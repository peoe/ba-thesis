\appendix
\chapter{Aditional Theorems}

The following theorems and corollaries will not be explicitly proven. We will however provide direct sources which include proofs.

\begin{theorem}[Riesz-Fr\'{e}chet; adapted from~\cite{Ryn2008}, Theorem 5.2]\label{Rie-Fre}
	Let $(H, \inner{\cdot}{\cdot})$ be a Hilbert space. Then
	\[ \forall \varphi \in H'~\exists x \in H: \varphi(x) = \inner{\varphi}{x}. \]
\end{theorem}

\begin{corollary}
	In particular for $H = \ELL{2}[a, b], \inner{f}{g} := \int\limits_a^b f(x) \conj{g(x)}~dx$ and $H' = \ELL{2}[a,b]$, with $-\infty < a < b < \infty$, we obtain
	\[ \forall f \in \ELL{2}[a,b]~\exists g \in \ELL{2}[a,b]: f(g) = \inner{f}{g} = \int\limits_a^b f(x) \conj{g(x)}~dx. \]
\end{corollary}

\begin{theorem}[Riemann-Lebesgue; adapted from~\cite{Plo2018}, Lemma 1.27]\label{Rie-Leb}
	Let $f \in \ELL{1}[a,b]$ with $-\infty < a < b < \infty$. Then we have
	\[ \lim\limits_{\Ninf[k]} \int\limits_a^b f(x)\sin(xk)~dx = 0. \]
\end{theorem}

\begin{theorem}[Prokhorov; adapted from~\cite{San2015}, Box~1.4]\label{Prok}
	Suppose that ${(\mu_n)}_{\NinN}$ is a tight sequence of probability measures over a complete and separable metric space $X$, that is for every $\varepsilon > 0$ there exists a compact subset $K \subseteq X$ such that $\mu_n(X \setminus K) < \varepsilon$ for every \NinN. Then there exists $\mu \in \PM{X}$ and a subsequence ${(\mu_{n_k})}_{\NinN[k]}$ such that \weak[k \rightarrow{} \infty]{\mu_{n_k}}{\mu}. Conversely, every sequence \weak{\mu_n}{\mu} is necessarily tight.
\end{theorem}

\begin{theorem}[Arzela-Ascoli; adapted from~\cite{San2015}, Box 1.7]\label{Arz-Asc}
	Let $X$ be a compact metric space and \map[f_n]{X}{\R} be equicontinuous and equibounded for all \NinN. Equicontinuous meaning that for every $\epsilon > 0$, there exists a common $\delta > 0$ such that $|f_n(x) - f_n(y)| < \epsilon$ for all pairs $x, y \in X$ with $d(x, y) < \delta$ and for all \NinN. Equibounded referring to the property that there is a common constant $C$ with $|f_n(x)| \le C$ for all $x \in X$ and all \NinN.

	Then the sequence ${(f_n)}_{\NinN}$ admits a subsequence ${(f_{n_k})}_{\NinN[k]}$ which uniformly converges to a continuous function \map[f]{X}{\R}.
\end{theorem}

\begin{theorem}[Lebesgue, Dominated Convergence; adapted from~\cite{Bog2007}, Volume 1, Theorem 2.8.1]\label{Leb-DomCon}
	Suppose that $\mu$-integrable functions $f_n$ converge almost everywhere to a function $f$. If there exists a $\mu$-integrable function $\phi$ such that
	\[ |f_n| \le \phi \]
	a.e.~for every $n$, then the function $f$ is integrable and
	\[ \lim\limits_{\Ninf} \int\limits_X f_n~\D = \int\limits_X f~\D. \]
\end{theorem}