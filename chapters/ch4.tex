\chapter{Stochastic Estimation}\label{DualPlanEst}

In this part we will first introduce regularized optimal transport. This regularization allows for a duality result between the dual solution and the primal solution which we will then be used to estimate optimal plans and maps.

In the following, $\Omega_1 \subseteq \R^{n_1}$, $\Omega_2 \subseteq \R^{n_2}$ and $\Omega_1 \times \Omega_2 = \Omega \subseteq \R^{n_1 + n_2}$ will always be compact.

\todoinline{TODO:~figure out, what continuity $c$ is required to have}

\section{Regularized Optimal Transport}\label{RegOT}

There are multiple ways that regularization can take place, however we will focus on \textit{entropic regularization}. This form of regularization deals with positive integrable functions with finite entropy, i.e.~for $f \in \MF{\Omega}{\R}$
\[ E(f) := \int\limits_\Omega \vert f(z) \vert \log\left( \vert f(z) \vert \right)~\D[z] < \infty \]
with $0 \cdot \log(0) := 0$. From here we can define
\[ \LLL := \left\{ f \in \MF{\Omega}{\R} : \int\limits_\Omega \vert f(z) \vert \log^+(\vert f(z) \vert)~\D[z] < \infty \right\} \]
as a subset of all integrable functions with finite entropy, where we consider $\log^+(x) := \max \{ 0, \log (x) \}$.

The first conclusion that can be reached via~\cite{Cla2021}, Proposition~2.1, is the equivalence of $f \in \MF{\Omega}{\RZero}$ with $E(f) < \infty$ and $f \in \LLL$.

The second conclusion regards the structure of \LLL{}. When considering the \textit{Luxemburg norm}
\[ \Vert f \Vert_\Phi := \inf \left\{ C > 0 : \int\limits_\Omega \Phi\left( \frac{\vert f(z) \vert}{C} \right)~\D[z] \le 1 \right\} \]
for measurable functions $f$, we denote $\Orlicz{\Phi} := (\mathcal{F}, \Vert \cdot \Vert_\Phi)$, where $\mathcal{F} := \{ f \in \MF{\Omega}{\R} : \Vert f \Vert_\Phi < \infty \}$, the \textit{Orlicz space} for a certain function $\Phi$. Using $\Phi_{\log}(x) := x \log^+ (x)$, we have $\Orlicz{\Phi_{\log}} = \LLL$. In fact, by~\cite{Cla2021}, Theorem~2.5, \LLL{} is a Banach space with respect to its Luxemburg norm.

For a third conclusion we are interested in the dual space of \LLL{}. We first define
\[ \Phi_{\exp}(s) := \begin{cases}
	s, & 0 \le s \le 1 \\
	\exp(s - 1), & s > 1
\end{cases} \]
and then $\Lexp := \Orlicz{\Phi_{\exp}}$ in accordance to its corresponding Luxemburg norm $\Vert \cdot \Vert_{\Phi_{\exp}}$. As shown in~\cite{Cla2021}, Proposition~2.7, if $\Omega$ has finite Lebesgue measure (which it has by $\Omega \subseteq \R^{n_1 + n_2}$ being compact and thus bounded), then ${(\LLL)}^* = \Lexp$.

\begin{definition}[Regularized Problem; adapted from~\cite{Cla2021}, (P), (D) and Proposition~3.1]\label{RegProbs}
	We once again consider our starting measures $\mu \in \PM{\Omega_1}$ and $\nu \in \PM{\Omega_2}$, and a cost function $\map[c]{\Omega}{\RZero \cup \{ \infty \}}$. The \textbf{Regularized Problem} for some $\varepsilon > 0$ is defined as
	\[ \inf\limits_{\gamma \in \TP{\mu}{\nu}} \int\limits_\Omega c(x, y)~\Dx[x, y]{\gamma} + \varepsilon \int\limits_\Omega \gamma(x, y) \big( \log(\gamma(x, y)) - 1 \big)~\Dx[x, y]{}. \]
	Its \textbf{Predual Problem} for the same $\varepsilon > 0$ is defined as
	\[ \sup\limits_{\varphi \in \CF[b]{\Omega_1}, \psi \in \CF[b]{\Omega_2}} \int\limits_{\Omega_1} \varphi(x)~\Dx{\mu} + \int\limits_{\Omega_2} \psi(y)~\Dx[y]{\nu} - \varepsilon \int\limits_\Omega F_{\varepsilon}(x, y)~\Dx[x, y]{}, \]
	\[ \text{with } F_{\varepsilon}(x, y) := \exp\left( \frac{1}{\varepsilon} \big( \varphi(x) + \psi(y) - c(x, y) \big) \right). \]
	Just as in previous problem definition, we will use the abbreviations (RP), $\inf \text{(RP)}$, (PD) and $\sup \text{(PD)}$ for the problems and their solutions respectively. Further, by~\cite{Cla2021}, Proposition~3.1, $\inf \text{(RP)} = \sup \text{(PD)}$, and if $\sup \text{(PD)}$ is finite, (RP) admits a minimizer.
\end{definition}

With Theorem~\ref{KPAdmitPolishLSC} we already saw that an optimal transport plan exists for (KP). For (RP) however, we require more constraints on our marginal measures.

\begin{theorem}[Taken from~\cite{Cla2021}, Theorem~3.3]\label{RegProbAdmitLLL}
	The problem$\text{\normalfont\ (RP)}$ admits a minimizer $\gamma \in \TP{\mu}{\nu}$ if and only if $\mu \in \LLL[\Omega_1]$ and $\nu \in \LLL[\Omega_2]$. In this case, $\gamma \in \LLL$ and $\gamma$ is unique.
\end{theorem}

\begin{proof}
	Cf.~the proof of Theorem~3.3 in~\cite{Cla2021}.
\end{proof}

As we have previously seen, the regularization term is finite if and only if $\gamma \in \LLL$. By considering (RP) over \LLL{} instead, we can also derive a dual problem over $\Lexp[\Omega_1] \times \Lexp[\Omega_2]$. Following the procedure from~\cite{Cla2021}, Section~4, we define
\[ \Phi(s) := \begin{cases}
	\infty, & s < 0 \\
	s, & 0 \le s \le 1 \\
	\exp(s - 1), & s > 1
\end{cases}, \Psi(s) := \begin{cases}
	- \infty, & s \le 0 \\
	\log(s), & 0 < s \le 1 \\
	s - 1, & s > 1
\end{cases}, \]
\[ u_1 := \begin{cases}
	\exp \left( \frac{\varphi}{\varepsilon} \right), & \varphi \le 0 \\
	\frac{\varphi}{\varepsilon} + 1, & \varphi > 0
\end{cases} \text{, and } u_2 := \begin{cases}
	\exp \left( \frac{\psi}{\varepsilon} \right), & \psi \le 0 \\
	\frac{\psi}{\varepsilon}, & \psi > 0
\end{cases} \]
and obtain $\varphi = \varepsilon \log \big( \Phi(u_1) \big) = \varepsilon \Psi(u_1) \text{ and } \psi = \varepsilon \log \big( \Phi(u_2) \big) = \varepsilon \Psi(u_2)$. When put back into (PD), this gives us
\[ \int\limits_{\Omega_1} \varphi(x)~\Dx{\mu} + \int\limits_{\Omega_2} \psi(y)~\Dx[y]{\nu} - \varepsilon \int\limits_{\Omega} \exp \left( \frac{\varphi(x) + \psi(y) - c(x, y)}{\varepsilon} \right)~\Dx[x, y]{} \]
\[ = \varepsilon \int\limits_{\Omega_1} \Psi(u_1(x))~\Dx{\mu} + \varepsilon \int\limits_{\Omega_2} \Psi(u_2(y))~\Dx[y]{\nu} - \varepsilon \int\limits F_{\varepsilon}^{\Phi}(x, y)~\Dx[x, y]{}, \]
with $F_{\varepsilon}^{\Phi}(x, y) := \Phi\big( u_1(x) \big) \Phi\big( u_2(y) \big) \exp\left( \frac{-c(x, y)}{\varepsilon} \right)$.

We note that $\Phi = \Phi_{\exp}$ on \RZero, suggesting an optimization over ${(u_1, u_2)} \in \Lexp[\Omega_1] \times \Lexp[\Omega_2]$ with $u_1, u_2 \ge 0$ (as the problem would be infeasible otherwise).

\begin{definition}[Regularized Dual; adapted from~\cite{Cla2021}, ($\text{D}_{\exp}$)]\label{RegDualProb}
	Let $\mu, \nu, \varepsilon$ as well as $c$ be the same as in Definition~\ref{RegProbs}. We consider the set $\mathcal{F} := \{ (f, g) \in \Lexp[\Omega_1] \times \Lexp[\Omega_2] : f, g \ge 0 \}$ and define the \textbf{Dual Problem}
	\[ \sup\limits_{\mathcal{F}} \varepsilon \int\limits_{\Omega_1} \Psi(u_1(x))~\Dx{\mu} + \varepsilon \int\limits_{\Omega_2} \Psi(u_2(y))~\Dx[y]{\nu} - \varepsilon \int\limits_{\Omega} F_{\varepsilon}^{\Phi}(x, y)~\Dx[x, y]{}. \]
	We label this problem accordingly as (RD) and its supremum as $\sup \text{(RD)}$. The restriction on $\mathcal{F}$ ensures that all integrals are well defined.
\end{definition}

\begin{theorem}[Adapted from~\cite{Cla2021}, Theorem~4.6]\label{RegDualAdmit}
	The problem$\text{\normalfont\ (RD)}$ admits a solution $(\bar{u}_1, \bar{u}_2) \in \Lexp[\Omega_1] \times \Lexp[\Omega_2]$.
\end{theorem}

\begin{proof}
	Cf.~the proof of Theorem~4.6 in~\cite{Cla2021}.
\end{proof}

Using the previously seen, we can now resubstitute $\bar{\phi} = \varepsilon \Psi(\bar{u}_1)$ as well as $\bar{\psi} = \varepsilon \Psi(\bar{u}_2)$ from our dual solutions. We can further assume $\bar{u}_1 > 0$ $\mu$-a.e.\ and $\bar{u}_2 > 0$ $\nu$-a.e., as otherwise $\int \Psi(\bar{u}_1)~\D + \int \Psi(\bar{u}_2)~\D[\nu] = -\infty$, turning the problem infeasible. The pair $(\bar{\phi}, \bar{\psi})$ however is not guaranteed to be admissible in (PD), i.e.~$\CF[b]{\Omega_1} \times \CF[b]{\Omega_2}$, as even for $\bar{u}_1, \bar{u}_2 > 0$ we do not necessarily get bounds on $\bar{\phi}$ and $\bar{\psi}$, since $\Psi(x) \rightarrow -\infty$ for $x \rightarrow 0$.

If we instead restrict ourselves to the assumptions from Theorem~\ref{RegProbAdmitLLL}, we obtain strong duality between (RP) and (RD).

\begin{theorem}[Adapted from~\cite{Cla2021}, Proposition~4.7]\label{RegStrongDualityLLL}
	We consider $\mu \in \LLL[\Omega_1], \nu \in \LLL[\Omega_2]$ and $c \in \CF{\Omega}$. Then$\text{\normalfont\ (RP)}$ admits a solution $\bar{\gamma} \in \LLL$,$\text{\normalfont\ (RD)}$ admits a solution $(\bar{u}_1, \bar{u}_2) \in \Lexp[\Omega_1] \times \Lexp[\Omega_2]$, and $\sup\!\text{\normalfont\,(RD)} = \inf\!\text{\normalfont\,(RP)}$ holds.
\end{theorem}

\begin{proof}
	\todoinline{TODO:~write exhaustive proof! cf.~\cite{Cla2021}, proof of theorem~4.8}
\end{proof}

\begin{corollary}[Conditions on Dual Optimality; adapted from~\cite{Cla2021}, Theorem~4.8]\label{RegOptCond}
	Let $\mu \in \LLL[\Omega_1], \nu \in \LLL[\Omega_2]$ and $c \in \CF{\Omega}$, as in Theorem~\ref{RegStrongDualityLLL}. Then for $\mu$-a.e.\ $x \in \Omega_1$ and $\nu$-a.e.\ $y \in \Omega_2$ all dual solutions $(\bar{u}_1, \bar{u}_2) \in \Lexp[\Omega_1] \times \Lexp[\Omega_2]$ of$\text{\normalfont\ (RD)}$ satisfy
	\[ \mu(x) = \Phi\big( \bar{u}_1(x) \big) \int\limits_{\Omega_2} \Phi\big( \bar{u}_2(y) \big) \exp \left( \frac{- c(x, y)}{\varepsilon} \right)\D[y], \]
	and
	\[ \nu(y) = \Phi\big( \bar{u}_2(y) \big) \int\limits_{\Omega_1} \Phi\big( \bar{u}_1(x) \big) \exp \left( \frac{- c(x, y)}{\varepsilon} \right)\D[x]. \]
	Furthermore, a solution $\bar{\gamma} \in \LLL$ of$\text{\normalfont\ (RP)}$ is defined by
	\[ \bar{\gamma}(x, y) = \Phi\big( \bar{u}_1(x) \big) \Phi\big( \bar{u}_2(y) \big) \exp\left( \frac{- c(x, y)}{\varepsilon} \right). \]
\end{corollary}

\begin{proof}
	Cf.~the proof of Theorem~4.8 in~\cite{Cla2021}.
\end{proof}

As~\cite{Cla2021}, Remark~4.9, points out, we can use this system to derive the \textit{Sinkhorn algorithm}. We make use of the first two equations to iteratively approach optimal dual variables, and apply the last equation to obtain an optimal transport plan. For a starting value $u_2^0 \in \Lexp[\Omega_2]$ we define
\[ T_1^{n + 1}(x) := \Phi^{-1} \left( \frac{\mu(x)}{\int_{\Omega_2} \Phi\big( u_2^n(y) \big) \exp\left( \frac{-c(x, y)}{\varepsilon} \right)\D[y]} \right), \]
and
\[ T_2^{n + 1}(y) := \Phi^{-1} \left( \frac{\nu(y)}{\int_{\Omega_2} \Phi\big( u_1^{n + 1}(x) \big) \exp\left( \frac{-c(x, y)}{\varepsilon} \right)\D[x]} \right) \]
for all \NinN, $(x, y) \in \Omega$.
\begin{algorithm}\label{SinkhornAlg}
	\caption{Sinkhorn Algorithm}
	\KwResult{Optimal dual variables $\bar{u}_1, \bar{u}_2$}
	Initialization: $\mu \in \PM{\Omega_1}, \nu \in \PM{\Omega_2}, (x, y) \in \Omega, \varepsilon > 0$, cost function $c$, starting variable $u_2^0 \in \Lexp[\Omega_2]$\;
	$u_1^1(x) := T_1^1(x)$\;
	$u_2^1(y) := T_2^1(y)$\;
	$n := 1$\;
	\While{$\text{\normalfont{}not converged}$}{
		$u_1^{n + 1}(x) := T_1^{n + 1}(x)$\;
		$u_2^{n + 1}(x) := T_2^{n + 1}(y)$\;
		$n := n + 1$\;
	}
\end{algorithm}

Stopping after $N - 1$ iterations, we get
\[ \gamma_N(x, y) = \Phi\big( u_1^N(x) \big) \Phi\big( u_2^N(y) \big) \exp\left( \frac{- c(x, y)}{\varepsilon} \right) \]
as an approximate value of transport plan for (RP) at a predetermined $(x, y)$.

\section{Optimal Transport Plan Estimation}\label{OTPlanEst}

As~\cite{Seg2018}, Section~1, Large-scale OT, points out, the Sinkhorn algorithm is only useful when considering discrete measures over small samples as each of its iterations has $\mathcal{O}(n^2)$ complexity. Hence, another approach is required when the underlying measures are continuous or the individual sample sizes get too large.

\todoinline{TODO:~Stochastic estimation, starting from regularized OT~\cite{Seg2018}, Theorem 3.1 and Algorithm 1}

\section{Optimal Transport Map Estimation}\label{OTMapEst}

\todoinline{TODO:~Mapping estimation, starting with~\cite{Seg2018}, Definition 4.1}