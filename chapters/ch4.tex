\chapter{Dual Problems and Transport Plan Estimation}\label{DualPlanEst}

\section{The Kantorovich Dual}\label{KanDual}

As~\cite{San2015} points out at the beginning of Section~1.2, the problem (KP) is a linear problem with convex constraints. As such, we are going to investigate the \textit{dual problem} to attain further results. One of these results will be regarding \textit{regularized optimal transport}.
\todoinline{why linear? why convex?}

\todoinline{TODO:~Rewrite section intro when seciton is done}

We will be expressing the dual problem of (KP) in terms of continuous bounded functions. For this it is useful to restate the constraint $\gamma \in \TP{\mu}{\nu}$ in a different manner. Firstly, we notice that for a general positive measure $\gamma \in \M[+]{X \times Y}$ the equation
\[ \sup\limits_{\varphi \in \CF[b]{X}, \psi \in \CF[b]{Y}} \int\limits_X \varphi(x)~\Dx{\mu} + \int\limits_Y \psi(y)~\Dx[y]{\nu} - \int\limits_{X \times Y} \varphi(x) + \psi(y)~\Dx[x, y]{\gamma} \]
is $0$, if $\gamma \in \TP{\mu}{\nu}$, and $\infty$ otherwise. This effectively constitutes a restatement of our constraint if we add it to the original formulation of (KP): if the constraint is fulfilled, nothing will be changed, and if $\gamma$ is not a transport plan, (KP) will stay infeasible. For ease of notation, we will be using the notation $(\varphi \oplus \psi)(x, y) := \varphi(x) + \psi(y)$. Exchanging the $\inf$ and the $\sup$ in the modified (KP), we obtain
\[ \sup\limits_{\varphi \in \CF[b]{X}, \psi \in \CF[b]{Y}} \int\limits_{X} \varphi~\D + \int\limits_{Y} \psi~\D[\nu] + \inf\limits_{\gamma \in \M[+]{X \times Y}} \int\limits_{X \times Y} c - \varphi \oplus \psi~\D[\gamma] \]
Secondly, we want to restate the above infimum in $\gamma$ as a constraint on $\varphi$ and $\psi$. We get
\[ \inf\limits_{\gamma \in \M[+]{X \times Y}} \int\limits_{X \times Y} c - \varphi \oplus \psi~\D[\gamma] = 
\begin{cases}
	0, & \varphi \oplus \psi \le c \text{ on } X \times Y \\
	- \infty, & \text{else}
\end{cases}. \]
This equation holds, as~\cite{San2015} points out, since if $\varphi \oplus \psi > c$ somewhere in $X \times Y$, we can choose measure $\gamma$ supported on that region with masses tending to $\infty$. 

We can now formulate the dual problem to the Kantorovich Problem:

\begin{definition}[Kantorovich Dual Problem]\label{DualProb}
	Let $\mu \in \PM{X}, \nu \in \PM{Y}$ be our usual measures and $\map[c]{X \times Y}{\RZero{} \cup{} \{ \infty \}}$ be a cost function. Then we define the \textbf{Kantorovich Dual Problem} as
	\begin{equation}\label{DPEq}
		\sup \left\{ \int\limits_{X} \varphi~\D + \int\limits_{Y} \psi~\D[\nu] : \varphi \in \CF[b]{X}, \psi \in \CF[b]{Y}, \varphi \oplus \psi \le c \right\}.
	\end{equation}
	In analogy to Definitions~\ref{MongeProb}~and~\ref{KanProb} we will be using the abbreviation (DP) to describe the problem itself and $\inf \text{(DP)}$ to denominate its infimum.
\end{definition}

With the formulation of (DP) and the previous characterization of the constraint $\gamma \in \TP{\mu}{\nu}$ we can now reach a direct result for the relationship between the primal and the dual poblems. For all admissible $\gamma \in \TP{\mu}{\nu}$ (i.e.~the constraint from (KP)) and $\varphi \in \CF[b]{X}, \psi \in \CF[b]{Y}$ with $\varphi \oplus \psi \le c$ (i.e.~the constraint from (DP)) we have
\[ \sup\limits_{\varphi \in \CF[b]{X}, \psi \in \CF[b]{Y}} \int\limits_{X} \varphi(x)~\Dx{\mu} + \int\limits_{Y} \psi(y)~\Dx[y]{\nu}  \]
\[ = \sup\limits_{\varphi \in \CF[b]{X}, \psi \in \CF[b]{Y}} \int\limits_{X \times Y} (\varphi \oplus \psi)(x, y)~\Dx[x, y]{\gamma} \le \int\limits_{X \times Y} c(x, y)~\Dx[x, y]{\gamma}. \]
As the left hand side of this equality is just a constant, we can further consider the infimum over all $\gamma \in \TP{\mu}{\nu}$ and thus obtain $\sup \text{(DP)} \le \inf \text{(KP)}$.

So far, we do not know if the $\sup$ of the left hand side does exist. To handle this problem, we will be further transforming (DP) using so called \textit{$c$-transforms}, or \textit{$c$-conjugate functions}. With these transformations it will then be possible to formulate (DP) as a optimization problem over just one variable.

\begin{definition}[$c$-Transform, $\bar{c}$-Transform, $c$-Concavity, $\bar{c}$-Concavity; taken from~\cite{San2015}, Definition 1.10]\label{cTrafo}
	Given two functions \map[\chi]{X}{\Rbar} and \map[\zeta]{Y}{\Rbar}, as well as $\map[c]{X \times Y}{\RZero{} \cup{} \{ \infty \}}$ a cost function, we define the \textbf{$\text{\textbf{\textit{c}}}$-transform} of $\chi$ by
	\[ \map[\chi^c]{Y}{\Rbar},~y \mapsto \inf\limits_{x \in X} c(x, y) - \chi(x) \]
	and the \textbf{$\bar{\text{\textbf{\textit{c}}}}$-transform} of $\zeta$ by
	\[ \map[\zeta^{\bar{c}}]{X}{\Rbar},~x \mapsto \inf\limits_{y \in Y} c(x, y) - \zeta(y). \]
	We further define the notion for a function \map[\psi]{Y}{\Rbar} to be \textbf{$\bar{\text{\textbf{\textit{c}}}}$-concave} if there exists a function \map[\chi]{Y}{\Rbar}, such that $\psi = \chi^c$. The set of all $\bar{c}$-concave functions over $Y$ will be denoted by \CBConc{Y}. Analogously, a function \map[\varphi]{X}{\Rbar} is said to be \textbf{$\text{\textbf{\textit{c}}}$-concave} if there exists a function \map[\zeta]{X}{\Rbar}, such that $\varphi = \zeta^{\bar{c}}$, and the set of all such $c$-concave functions over $X$ will accordingly be described by \CConc{X}.
\end{definition}

\todoinline{TODO:~Go via~\cite{San2015} towards convergence of optimal plans --- though possibly a shorter version.~\cite{San2015}, Theorem 1.39 $\rightarrow$ relax to Theorem 1.42 $\rightarrow$ motivate~\cite{Seg2018}, Theorem 1}

\section{Optimal Transport Plan Estimation}\label{OTPlanEst}

\todoinline{TODO:~Stochastic estimation, starting from regularized OT~\cite{San2015}, Theorem 1 and Algorithm 1}