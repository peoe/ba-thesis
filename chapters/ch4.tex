\chapter{Stochastic Estimation}\label{DualPlanEst}

In this part we will first introduce regularized optimal transport. This regularization allows for a duality result between the dual solution and the primal solution which we will then be used to estimate optimal plans and maps.

In the following, $\Omega_1 \subseteq \R^{n_1}$, $\Omega_2 \subseteq \R^{n_2}$ and $\Omega_1 \times \Omega_2 = \Omega \subseteq \R^{n_1 + n_2}$ will always be compact.

\todoinline{TODO:~figure out, what continuity $c$ is required to have}

\section{Regularized Optimal Transport}\label{RegOT}

There are multiple ways that regularization can take place, however we will focus on \textit{entropic regularization}. This form of regularization deals with positive integrable functions with finite entropy, i.e.~for $f \in \MF{\Omega}{\R}$
\[ E(f) := \int\limits_\Omega \vert f(z) \vert \log\left( \vert f(z) \vert \right)~\D[z] < \infty \]
with $0 \cdot \log(0) := 0$. From here we can define
\[ \LLL := \left\{ f \in \MF{\Omega}{\R} : \int\limits_\Omega \vert f(z) \vert \log^+(\vert f(z) \vert)~\D[z] < \infty \right\} \]
as a subset of all integrable functions with finite entropy, where we consider $\log^+(x) := \max \{ 0, \log (x) \}$.

The first conclusion that can be reached via~\cite{Cla2021}, Proposition~2.1, is the equivalence of $f \in \MF{\Omega}{\RZero}$ with $E(f) < \infty$ and $f \in \LLL$.

The second conclusion regards the structure of \LLL{}. When considering the \textit{Luxemburg norm}
\[ \Vert f \Vert_\Phi := \inf \left\{ C > 0 : \int\limits_\Omega \Phi\left( \frac{\vert f(z) \vert}{C} \right)~\D[z] \le 1 \right\} \]
for measurable functions $f$, we denote $\Orlicz{\Phi} := (\mathcal{F}, \Vert \cdot \Vert_\Phi)$, where $\mathcal{F} := \{ f \in \MF{\Omega}{\R} : \Vert f \Vert_\Phi < \infty \}$, the \textit{Orlicz space} for a certain function $\Phi$. Using $\Phi_{\log}(x) := x \log^+ (x)$, we have $\Orlicz{\Phi_{\log}} = \LLL$. In fact, by~\cite{Cla2021}, Theorem~2.5, \LLL{} is a Banach space with respect to its Luxemburg norm.

For a third conclusion we are interested in the dual space of \LLL{}. We first define
\[ \Phi_{\exp}(s) := \begin{cases}
	s, & 0 \le s \le 1 \\
	\exp(s - 1), & s > 1
\end{cases} \]
and then $\Lexp := \Orlicz{\Phi_{\exp}}$ in accordance to its corresponding Luxemburg norm $\Vert \cdot \Vert_{\Phi_{\exp}}$. As shown in~\cite{Cla2021}, Proposition~2.7, if $\Omega$ has finite Lebesgue measure (which it has by $\Omega \subseteq \R^{n_1 + n_2}$ being compact and thus bounded), then ${(\LLL)}^* = \Lexp$.

\begin{definition}[Regularized Problem; adapted from~\cite{Cla2021}, (P), (D) and Proposition~3.1]\label{RegProbs}
	We once again consider our starting measures $\mu \in \PM{\Omega_1}$ and $\nu \in \PM{\Omega_2}$, and a cost function $\map[c]{\Omega}{\RZero \cup \{ \infty \}}$. The \textbf{Regularized Problem} for some $\varepsilon > 0$ is defined as
	\[ \inf\limits_{\gamma \in \TP{\mu}{\nu}} \int\limits_\Omega c(x, y)~\Dx[x, y]{\gamma} + \varepsilon \int\limits_\Omega \gamma(x, y) \big( \log(\gamma(x, y)) - 1 \big)~\Dx[x, y]{}. \]
	Its \textbf{Predual Problem} for the same $\varepsilon > 0$ is defined as
	\[ \sup\limits_{\varphi \in \CF[b]{\Omega_1}, \psi \in \CF[b]{\Omega_2}} \int\limits_{\Omega_1} \varphi(x)~\Dx{\mu} + \int\limits_{\Omega_2} \psi(y)~\Dx[y]{\nu} - \varepsilon \int\limits_\Omega F_{\varepsilon}(x, y)~\Dx[x, y]{}, \]
	\[ \text{with } F_{\varepsilon}(x, y) := \exp\left( \frac{1}{\varepsilon} \big( \varphi(x) + \psi(y) - c(x, y) \big) \right). \]
	Just as in previous problem definition, we will use the abbreviations (RP), $\inf \text{(RP)}$, (PD) and $\sup \text{(PD)}$ for the problems and their solutions respectively. Further, by~\cite{Cla2021}, Proposition~3.1, $\inf \text{(RP)} = \sup \text{(PD)}$, and if $\sup \text{(PD)}$ is finite, (RP) admits a minimizer.
\end{definition}

With Theorem~\ref{KPAdmitPolishLSC} we already saw that an optimal transport plan exists for (KP). For (RP) however, we require more constraints on our marginal measures.

\begin{theorem}[Taken from~\cite{Cla2021}, Theorem~3.3]\label{RegProbAdmitLLL}
	The problem$\text{\normalfont\ (RP)}$ admits a minimizer $\gamma \in \TP{\mu}{\nu}$ if and only if $\mu \in \LLL[\Omega_1]$ and $\nu \in \LLL[\Omega_2]$. In this case, $\gamma \in \LLL$ and $\gamma$ is unique.
\end{theorem}

\begin{proof}
	Cf.~the proof of Theorem~3.3 in~\cite{Cla2021}.
\end{proof}

As we have previously seen, the regularization term is finite if and only if $\gamma \in \LLL$. By considering (RP) over \LLL{} instead, we can also derive a dual problem over $\Lexp[\Omega_1] \times \Lexp[\Omega_2]$. Following the procedure from~\cite{Cla2021}, Section~4, we define
\[ \Phi(s) := \begin{cases}
	\infty, & s < 0 \\
	s, & 0 \le s \le 1 \\
	\exp(s - 1), & s > 1
\end{cases}, \Psi(s) := \begin{cases}
	- \infty, & s \le 0 \\
	\log(s), & 0 < s \le 1 \\
	s - 1, & s > 1
\end{cases}, \]
\[ u_1 := \begin{cases}
	\exp \left( \frac{\varphi}{\varepsilon} \right), & \varphi \le 0 \\
	\frac{\varphi}{\varepsilon} + 1, & \varphi > 0
\end{cases} \text{, and } u_2 := \begin{cases}
	\exp \left( \frac{\psi}{\varepsilon} \right), & \psi \le 0 \\
	\frac{\psi}{\varepsilon}, & \psi > 0
\end{cases} \]
and obtain $\varphi = \varepsilon \log \big( \Phi(u_1) \big) = \varepsilon \Psi(u_1) \text{ and } \psi = \varepsilon \log \big( \Phi(u_2) \big) = \varepsilon \Psi(u_2)$. When put back into (PD), this gives us
\[ \int\limits_{\Omega_1} \varphi(x)~\Dx{\mu} + \int\limits_{\Omega_2} \psi(y)~\Dx[y]{\nu} - \varepsilon \int\limits_{\Omega} \exp \left( \frac{\varphi(x) + \psi(y) - c(x, y)}{\varepsilon} \right)~\Dx[x, y]{} \]
\[ = \varepsilon \int\limits_{\Omega_1} \Psi(u_1(x))~\Dx{\mu} + \varepsilon \int\limits_{\Omega_2} \Psi(u_2(y))~\Dx[y]{\nu} - \varepsilon \int\limits F_{\varepsilon}^{\Phi}(x, y)~\Dx[x, y]{}, \]
with $F_{\varepsilon}^{\Phi}(x, y) := \Phi\big( u_1(x) \big) \Phi\big( u_2(y) \big) \exp\left( \frac{-c(x, y)}{\varepsilon} \right)$.

We note that $\Phi = \Phi_{\exp}$ on \RZero, suggesting an optimization over ${(u_1, u_2)} \in \Lexp[\Omega_1] \times \Lexp[\Omega_2]$ with $u_1, u_2 \ge 0$ (as the problem would be infeasible otherwise).

\begin{definition}[Regularized Dual; adapted from~\cite{Cla2021}, ($\text{D}_{\exp}$)]\label{RegDualProb}
	Let $\mu, \nu, \varepsilon$ as well as $c$ be the same as in Definition~\ref{RegProbs}. We consider the set $\mathcal{F} := \{ (f, g) \in \Lexp[\Omega_1] \times \Lexp[\Omega_2] : f, g \ge 0 \}$ and define the \textbf{Dual Problem}
	\[ \sup\limits_{\mathcal{F}} \varepsilon \int\limits_{\Omega_1} \Psi(u_1(x))~\Dx{\mu} + \varepsilon \int\limits_{\Omega_2} \Psi(u_2(y))~\Dx[y]{\nu} - \varepsilon \int\limits_{\Omega} F_{\varepsilon}^{\Phi}(x, y)~\Dx[x, y]{}. \]
	We label this problem accordingly as (RD) and its supremum as $\sup \text{(RD)}$. The restriction on $\mathcal{F}$ ensures that all integrals are well defined.
\end{definition}

\begin{theorem}[Adapted from~\cite{Cla2021}, Theorem~4.6]\label{RegDualAdmit}
	The problem$\text{\normalfont\ (RD)}$ admits a solution $(\bar{u}_1, \bar{u}_2) \in \Lexp[\Omega_1] \times \Lexp[\Omega_2]$.
\end{theorem}

\begin{proof}
	Cf.~the proof of Theorem~4.6 in~\cite{Cla2021}.
\end{proof}

Using the previously seen, we can now resubstitute $\bar{\phi} = \varepsilon \Psi(\bar{u}_1)$ as well as $\bar{\psi} = \varepsilon \Psi(\bar{u}_2)$ from our dual solutions. We can further assume $\bar{u}_1 > 0$ $\mu$-a.e.\ and $\bar{u}_2 > 0$ $\nu$-a.e., as otherwise $\int \Psi(\bar{u}_1)~\D + \int \Psi(\bar{u}_2)~\D[\nu] = -\infty$, turning the problem infeasible. The pair $(\bar{\phi}, \bar{\psi})$ however is not guaranteed to be admissible in (PD), i.e.~$\CF[b]{\Omega_1} \times \CF[b]{\Omega_2}$, as even for $\bar{u}_1, \bar{u}_2 > 0$ we do not necessarily get bounds on $\bar{\phi}$ and $\bar{\psi}$, since $\Psi(x) \rightarrow -\infty$ for $x \rightarrow 0$.

If we instead restrict ourselves to the assumptions from Theorem~\ref{RegProbAdmitLLL}, we obtain strong duality between (RP) and (RD).

\begin{theorem}[Adapted from~\cite{Cla2021}, Proposition~4.7]\label{RegStrongDualityLLL}
	We consider $\mu \in \LLL[\Omega_1], \nu \in \LLL[\Omega_2]$ and $c \in \CF{\Omega}$. Then$\text{\normalfont\ (RP)}$ admits a solution $\bar{\gamma} \in \LLL$,$\text{\normalfont\ (RD)}$ admits a solution $(\bar{u}_1, \bar{u}_2) \in \Lexp[\Omega_1] \times \Lexp[\Omega_2]$, and $\sup\!\text{\normalfont\,(RD)} = \inf\!\text{\normalfont\,(RP)}$ holds.
\end{theorem}

\begin{proof}
	\todoinline{TODO:~write exhaustive proof! cf.~\cite{Cla2021}, proof of theorem~4.8}
\end{proof}

\begin{corollary}[Conditions on Dual Optimality; adapted from~\cite{Cla2021}, Theorem~4.8]\label{RegOptCond}
	Let $\mu \in \LLL[\Omega_1], \nu \in \LLL[\Omega_2]$ and $c \in \CF{\Omega}$, as in Theorem~\ref{RegStrongDualityLLL}. Then for $\mu$-a.e.\ $x \in \Omega_1$ and $\nu$-a.e.\ $y \in \Omega_2$ all dual solutions $(\bar{u}_1, \bar{u}_2) \in \Lexp[\Omega_1] \times \Lexp[\Omega_2]$ of$\text{\normalfont\ (RD)}$ satisfy
	\[ \mu(x) = \Phi\big( \bar{u}_1(x) \big) \int\limits_{\Omega_2} \Phi\big( \bar{u}_2(y) \big) \exp \left( \frac{- c(x, y)}{\varepsilon} \right)\D[y], \]
	and
	\[ \nu(y) = \Phi\big( \bar{u}_2(y) \big) \int\limits_{\Omega_1} \Phi\big( \bar{u}_1(x) \big) \exp \left( \frac{- c(x, y)}{\varepsilon} \right)\D[x]. \]
	Furthermore, a solution $\bar{\gamma} \in \LLL$ of$\text{\normalfont\ (RP)}$ is defined by
	\[ \bar{\gamma}(x, y) = \Phi\big( \bar{u}_1(x) \big) \Phi\big( \bar{u}_2(y) \big) \exp\left( \frac{- c(x, y)}{\varepsilon} \right). \]
\end{corollary}

\begin{proof}
	Cf.~the proof of Theorem~4.8 in~\cite{Cla2021}.
\end{proof}

As~\cite{Cla2021}, Remark~4.9, points out, we can use this system to derive the \textit{Sinkhorn algorithm}. We make use of the first two equations to iteratively approach optimal dual variables, and apply the last equation to obtain an optimal transport plan. For a starting value $u_2^0 \in \Lexp[\Omega_2]$ we define
\[ T_1^{n + 1}(x) := \Phi^{-1} \left( \frac{\mu(x)}{\int_{\Omega_2} \Phi\big( u_2^n(y) \big) \exp\left( \frac{-c(x, y)}{\varepsilon} \right)\D[y]} \right), \]
and
\[ T_2^{n + 1}(y) := \Phi^{-1} \left( \frac{\nu(y)}{\int_{\Omega_2} \Phi\big( u_1^{n + 1}(x) \big) \exp\left( \frac{-c(x, y)}{\varepsilon} \right)\D[x]} \right) \]
for all \NinN, $(x, y) \in \Omega$.
\begin{algorithm}\label{SinkhornAlg}
	\caption{Sinkhorn Algorithm; adapted from~\cite{Cla2021}, Remark~4.9}
	\KwResult{Optimal dual variables $\bar{u}_1, \bar{u}_2$}
	\KwIn{$\mu \in \PM{\Omega_1}, \nu \in \PM{\Omega_2}, (x, y) \in \Omega, \varepsilon > 0$, cost function $c$, starting variable $u_2^0 \in \Lexp[\Omega_2]$}
	$n := 0$\;
	\While{$\text{\normalfont{}not converged}$}{
		$u_1^{n + 1}(x) := T_1^{n + 1}(x)$\;
		$u_2^{n + 1}(x) := T_2^{n + 1}(y)$\;
		$n := n + 1$\;
	}
\end{algorithm}

Stopping after $N - 1$ iterations, we get
\[ \gamma_N(x, y) = \Phi\big( u_1^N(x) \big) \Phi\big( u_2^N(y) \big) \exp\left( \frac{- c(x, y)}{\varepsilon} \right) \]
as an approximate value of transport plan for (RP) at a predetermined $(x, y)$. Resubstituting once more with $\exp\left( \frac{\varphi_N(x)}{\varepsilon} \right) = \Phi\big( u_1^N(x) \big)$ as well as $\exp\left( \frac{\psi_N(y)}{\varepsilon} \right) = \Phi\big( u_2^N(y) \big)$, we obtain
\begin{equation}\label{RegApprTransPlan}
	\gamma_N(x, y) = \exp\left( \frac{\varphi_N(x) + \psi_N(y) - c(x, y)}{\varepsilon} \right).
\end{equation}

This formulation will be used from now on, as it simplifies the following formulae, and by the proof of Theorem~\ref{RegStrongDualityLLL} $\sup\limits_{L_{\exp}} \text{(RD)}$ and $\sup\limits_{\mathcal{C}_b} \text{(PD)}$ coincide.

\section{Optimal Transport Plan Estimation}\label{OTPlanEst}

As~\cite{Seg2018}, Section~1, Large-scale OT, points out, the Sinkhorn algorithm is only useful when considering discrete measures over small samples as each of its iterations has $\mathcal{O}(n^2)$ complexity. Hence, another approach is required when the underlying measures are continuous or the individual sample sizes get too large.

The procedure taken by~\cite{Seg2018} improves on both shortcomings: Instead of only computing the approximated plan at a specified location $(x, y)$ a neural network is trained to model the continuous case and a finite vector is iteratively improved in the discrete case. To deal with larger samples, the input measures $\mu$ and $\nu$ are interpreted as probability distributions and individual samples drawn from them for each iteration.

In order to state the algorithm, we need to adapt the previous definition of $F_\varepsilon$ slightly to include the dual variables:
\[ F_\varepsilon\big( \varphi(x), \psi(y) \big) := \exp\left( \frac{1}{\varepsilon} \big( \varphi(x) + \psi(y) - c(x, y) \big) \right). \]
When considering the continuous case, we further define $\nabla \varphi$ and $\nabla \psi$ as the gradients with respect to the parameters of the underlying neural networks, and the notions of $\partial_\varphi F_\varepsilon$ and $\partial_\psi F_\varepsilon$ in accordance with these gradients.

\begin{algorithm}\label{OTPlanEstAlg}
	\caption{Transport Plan Estimation; adapted from~\cite{Seg2018}, Algorithm~1}
	\KwResult{Estimates of optimal dual variables $\varphi, \psi$}
	\KwIn{$\mu \in \PM{\Omega_1}, \nu \in \PM{\Omega_2}, \varepsilon > 0$, cost function $c$, batch size $p$, learning rate $\rho$}
	Initialize $\varphi_0, \psi_0$\;
	$n := 0$\;
	\While{$\text{\normalfont{}not converged}$}{
		sample a batch $(x_1, \dots, x_p)$ from $\mu$\;
		sample a batch $(y_1, \dots, y_p)$ from $\nu$\;
		$\varphi_{n + 1} := \varphi_n + \rho \sum\limits_{i, j = 1}^p \Big( \nabla \varphi_n(x_i) - \varepsilon \partial_{\varphi_n} F_\varepsilon\big( \varphi_n(x_i), \psi_n(y_j) \big) \nabla\varphi_n(x_i) \Big)$\;
		$\psi_{n + 1} := \psi_n + \rho \sum\limits_{i, j = 1}^p \Big( \nabla \psi_n(x_i) - \varepsilon \partial_{\psi_n} F_\varepsilon\big( \varphi_n(x_i), \psi_n(y_j) \big) \nabla\psi_n(y_j) \Big)$\;
		$n := n + 1$\;
	}
\end{algorithm}

The most apparent advantage of this approach according to~\cite{Seg2018}, Section~3, Convergence rates and computational cost comparison, is that, in the case of $\mu$ and $\nu$ being discrete, the stochastic gradient descent method converges at a rate of $\mathcal{O}\left( n^{-\frac{1}{2}} \right)$, where $n$ is the iteration number, but only requires a cost of $\mathcal{O}(p^2)$ per iteration. In the case of both measures being continuous, the cost per iteration remains $\mathcal{O}(p^2)$, but convergence can only be ensured up to a stationary point.

Applying Equation~\ref{RegApprTransPlan}, we denote the approximated transport plan for a certain $\varepsilon > 0$ after $N$ steps of Algorithm~\ref{OTPlanEstAlg} by $\gamma_N^\varepsilon$. We are now interested in how this estimate for the regularized problem (RP) may result in a solution for the original probme (KP).\@~\cite{Seg2018}, Theorem~1, gives us a result.

\todoinline{TODO:~update the theorem statement to be more pretty\dots}

\begin{theorem}[Adapted from~\cite{Seg2018}, Theorem~1]\label{RelatRegSolOrigSolPlan}
	Let $X$ and $Y$ be complete metric spaces, $\Omega_1 \subseteq X, \Omega_2 \subseteq Y, \Omega := \Omega_1 \times \Omega_2$ be compact subsets, $\mu \in \PM{\Omega_1}, \nu \in \PM{\Omega_2}$ be our starting probability measures, \map[c]{\Omega}{\RZero} be a finite and continuous cost function, $\mu_n := \sum_{i = 1}^n a_i \delta_{x_i}, \nu_n := \sum_{j = 1}^n b_j \delta_{y_j}, x_i \in \Omega_1, a_i \ge 0, y_j \in \Omega_2, b_j \ge 0, 1 \le i, j \le n$, be discrete probability measures with \weak[]{\mu_n}{\mu} and \weak[]{\nu_n}{\nu} for \Ninf, and ${(\varepsilon_n)}_{\NinN}$ be a sequence converging to $0$ sufficiently fast with $\varepsilon_n > 0$ for all \NinN. Then for $\gamma_n^{\varepsilon_n}$, the solution of$\text{\normalfont\ (RP)}$ between $\mu_n$ and $\nu_n$ with $\varepsilon := \varepsilon_n$, there exists a subsequence ${(\gamma_{n_k}^{\varepsilon_{n_k}})}_{\NinN[k]}$ that weakly converges to a solution $\gamma$ of$\text{\normalfont\ (KP)}$ between $\mu$ and $\nu$ for \Ninf.
\end{theorem}

\begin{proof}
	We consider $\gamma_n$ the solution of (KP) between $\mu_n$ and $\nu_n$ with maximum entropy. According to \todo{Add reference to Villani, 2008, Theorem~5.20}, there now exists a subsequence which weakly converges to a solution $\gamma$ of the same problem between $\mu$ and $\nu$. We continue to label this subsequence as $\gamma_n$, and further take the same indices to form a subsequence from the solutions of (RP) between $\mu_n$ and $\nu_n$, labeling it $\gamma_n^{\varepsilon_n}$. The solutions of (RP) exist due to the discrete measures $\mu_n, \nu_n$ reducing the integral over $\Omega_1$ and $\Omega_2$ respectively to a finite sum over their masses $x_i, y_j$, hence allowing for the application of Theorem~\ref{RegProbAdmitLLL}. By expanding the following integral for $g \in \CF[b]{\Omega}$
	\[ \Bigg\lvert \int\limits_{\Omega} g~\D[\gamma_n^{\varepsilon_n}] - \int\limits_{\Omega} g~\D[\gamma] \Bigg\rvert = \Bigg\lvert \int\limits_{\Omega} g~\D[\gamma_n^{\varepsilon_n}] - \int\limits_{\Omega} g~\D[\gamma_n] \Bigg\rvert + \Bigg\lvert \int\limits_{\Omega} g~\D[\gamma_n] - \int\limits_{\Omega} g~\D[\gamma] \Bigg\rvert, \]
	we now only have to show that the left summand on the right hand side converges to $0$, as the right summand converges by the previous statement of \todo{Insert reference to Villani, 2008, Theorem~5.20}. As $\gamma_n^{\varepsilon_n}$ and $\gamma_n$ are solutions to optimal transport problems between discrete measures, we can replace the integral over $\Omega$ with sums over the masses of both measures and obtain
	\[ \Bigg\lvert \sum\limits_{i, j = 1}^n g(x_i, y_j) \gamma_n^{\varepsilon_n}(x_i, y_j) - \sum\limits_{i, j = 1}^n g(x_i, y_j) \gamma_n(x_i, y_j) \Bigg\rvert \le M_g \Vert \gamma_n^{\varepsilon_n} - \gamma_n \Vert_{1}^{n \times n}, \]
	where $\Vert \gamma_n^{\varepsilon_n} - \gamma_n \Vert_{1}^{n \times n} := \sum\limits_{i, j = 1}^n \big\lvert \gamma_n^{\varepsilon_n}(x_i, y_j) - \gamma_n(x_i, y_j) \big\rvert$ and $M_g := \max g(x_i, y_j)$ over $1 \le i, j \le n$. \todo{Add result by Cominetti and San Martin, 1994} shows that there exist $M_{c_n, \mu_n, \nu_n}, \lambda_{c_n, \mu_n, \nu_n} > 0$ such that
	\[ \Vert \gamma_n^{\varepsilon_n} - \gamma_n \Vert_1^{n \times n} \le M_{c_n, \mu_n, \nu_n} \exp\left( -\frac{\lambda_{c_n, \mu_n, \nu_n}}{\varepsilon_n} \right), \]
	with $c_n := {\big( c(x_i, y_j) \big)}_{i, j = 1}^n$. To show the convergence of the right hand side we can finally choose
	\[ \varepsilon_n = \frac{\lambda_{c_n, \mu_n, \nu_n}}{\ln(n\,M_{c_n, \mu_n, \nu_n})} \rightarrow 0, \Ninf. \]
\end{proof}

\section{Optimal Transport Map Estimation}\label{OTMapEst}

In the first part, specifically in Theorem~\ref{IndPlansDense}, we already saw, that the set of transport plans induced by transport maps is dense in the set of all transport maps \TP{\mu}{\nu}. We are now interested in finding a transport map from our previously computed estimation of an optimal transport plan, i.e.~a solution to (MP) between $\mu$ and $\nu$, as the handling of a mapping instead of a joint measure (probability distribution) is often advantageous. As we have already seen in Theorem~\ref{InfCoincide}, when $\mu$ is atomless, such a transport map is directly related to the accompanying transport plan. This suggests that we start with our weakly convergent subsequence and apply a transformation which will then result in a sequence weakly converging toward an optimal transport map.

The transformation~\cite{Seg2018} chose, is the so called \textit{barycentric projection}.

\begin{definition}[Barycentric Projection; adapted from~\cite{Seg2018}, Definition~1]\label{BarCentrProj}
	We consider a transport plan $\gamma \in \TP{\mu}{\nu}$ and a convex cost function $\map[d]{\Omega_2 \times \Omega_2}{\RZero \cup \{ \infty \}}$. Then the \textbf{barycentric projection} of $\gamma$ at $x \in \Omega_1$ is defined as
	\[ \bar{\gamma}(x) := \argmin\limits_{z \in \Omega_2} \int\limits_{\Omega_2} d(z, y)~\Dx[x, y]{\gamma}. \]
	If $d = \Vert x - y \Vert_2^2$, where $\Vert \cdot \Vert_2$ is the Euclidean norm, the barycentric projection has the closed form
	\[ \bar{\gamma}(x) = \int\limits_{\Omega_2} y~\Dx[x, y]{\gamma}. \]
\end{definition}

If $c(x, y) = \Vert x - y \Vert_2^2$, according to~\cite{Seg2018}, it can be further shown, that the barycentric projection of an optimal transport plan for (KP) is already an optimal transport map for (MP). For more general convex cost functions,~\cite{Seg2018} provide a method of approximating the barycentric projection, making use of a neural network once more, since this allows for the target map to be defined on all of $\Omega_1$. Making use of $F_\varepsilon$ as defined in Section~\ref{OTPlanEst} and a map $f_\theta$ parameterized as a deep neural network with the parameters $\theta$ the following algorithm is formulated.

\begin{algorithm}\label{OTMapEstAlg}
	\caption{Transport Map Estimation; adapted from~\cite{Seg2018}, Algorithm~2}
	\KwResult{Estimate of the barycentric projection $\bar{\gamma}^\varepsilon$}
	\KwIn{$\mu \in \PM{\Omega_1}, \nu \in \PM{\Omega_2}, \varepsilon > 0$, cost function $c$, convex cost function $d$, batch size $p$, learning rate $\rho$}
	Initialize $f_\theta$\;
	Compute estimates of optimal dual variables $\varphi, \psi$ using Algorithm~\ref{OTPlanEstAlg} with $\mu, \nu, \varepsilon, c, p$ and $\rho$\;
	\While{$\text{\normalfont{}not converged}$}{
		sample a batch $(x_1, \dots, x_p)$ from $\mu$\;
		sample a batch $(y_1, \dots, y_p)$ from $\nu$\;
		$\theta := \theta - \rho \sum\limits_{i, j = 1}^p F_\varepsilon\big( \varphi(x_i), \psi(y_j) \big) \nabla_\theta\,d\big( y_j, f_\theta(x_i) \big)$\;
	}
\end{algorithm}

This algorithm may also be used to compute the opposite barycentric projection $g$ with respect to a convex cost function $d$ on $\Omega_1 \times \Omega_1$. The last term in line~6 of Algorithm~\ref{OTMapEstAlg} should then be $d\big( g(y_j), x_i \big)$. This is due to the symmetry of the optimal transport problem, as~\cite{Seg2018} point out.

We are now interested in how the barycentric projection of our regularized transport plan relates to the map $f$ solving (MP). Similarly to Theorem~\ref{RelatRegSolOrigSolPlan}, we will consider $X = Y = \R^d$, $\Omega_1, \Omega_2 \subseteq \R^d, \Omega := \Omega_1 \times \Omega_2$ compact, a continuous measure $\mu \in \PM{\Omega_1}$ \todo{Check which preliminaries Villani, 2008, Corollary~9.3 entails!}, an arbitrary measure $\nu \in \PM{\Omega_2}$, a continuous cost function $\map[c]{\R^d \times \R^d}{\RZero \cup \{ \infty \}}$, as well as the discrete measures
\[ \mu_n := \frac{1}{n} \sum\limits_{i = 1}^n \delta_{x_i}, \nu_n := \frac{1}{n} \sum\limits_{j = 1}^n \delta_{y_j}, \]
with $x_i \in \Omega_1, y_j \in \Omega_2, 1 \le i, j \le n$, weakly converging to $\mu$ and $\nu$ respectively.

\begin{theorem}[Adapted from~\cite{Seg2018}, Theorem~2]\label{RelatRegSolOrigSolMap}
	Consider $X$, $Y$, $\Omega_1$, $\Omega_2$, $\Omega$, $\mu$, $\nu$, $c$, $\mu_n$, $\nu_n$ as described above. Assume that the solutions $\gamma_n$ to$\text{\normalfont\ (KP)}$ between $\mu_n$ and $\nu_n$ are unique for all \NinN. Further let ${(\varepsilon_n)}_{\NinN}$ be a sequence converging sufficiently fast to $0$ with $\varepsilon_n > 0$ for all \NinN, and $d$ be a convex cost function on $\Omega_2 \times \Omega_2$. Then there exists a subsequence ${\big(\bar{\gamma}^{\varepsilon_{n_k}}_{n_k}\big)}_{\NinN[k]}$ such that \weak[]{\push[\big(id_{\Omega_1}, \bar{\gamma}^{\varepsilon_{n_k}}_{n_k}\big)]{\mu_n}}{\push[(id_{\Omega_1}, f)]{\mu}} for \Ninf, where $f$ is the map solving$\text{\normalfont\ (MP)}$ between $\mu$ and $\nu$, $id_{\Omega_1}$ is the identity map on $\Omega_1$, and $\bar{\gamma}^{\varepsilon_{n_k}}_{n_k}$ is the barycentric projection of the solution $\gamma^{\varepsilon_{n_k}}_{n_k}$ of$\text{\normalfont\ (RP)}$ with respect to $d$.
\end{theorem}

\begin{proof}
	By \todo{Add reference to Villani, 2008, Corollary~9.3} there exists a Monge map between $\mu$ and $\nu$ due to $\mu$ being uniformly continuous on $\Omega_1$ and thus absolutely continuous with respect to the Lebesgue measure. Following the proof of~\cite{Seg2018}, Theorem~2, we have to show that for any Lipschitz continuous function $g$ over $\Omega$
	\[ \Biig\lvert \int\limits_{\Omega} g~\D[\push[id, \bar{\gamma}_n^{\varepsilon_n}]{\mu_n}] - \int\limits_{\Omega} g~\D[\push[id, \bar{\gamma}_n^]{\mu_n}] \Bigg\rvert \rightarrow 0 \]
	for \Ninf{} and $\varepsilon_n$ converging to $0$ sufficiently fast.
	\todoinline{TODO:~finish this proof! (\cite{Seg2018}, proof of Theorem~2)}
\end{proof}

\begin{corollary}[Taken from~\cite{Seg2018}, Corollary~1]\label{RelatRegOrigCor}
	With the same assumptions as for Theorem~\ref{RelatRegSolOrigSolMap}, \weak[]{\push[\big(\bar{\gamma}^{\varepsilon_{n_k}}_{n_k}\big)]{\mu_{n_k}}}{\nu} for \Ninf.
\end{corollary}

\begin{proof}
	\todoinline{TODO:~finish this proof! (\cite{Seg2018}, proof of Corollary~1)}
\end{proof}