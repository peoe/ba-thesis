\chapter{Stochastic Estimation}\label{DualPlanEst}

In this part we will first introduce regularized optimal transport. This regularization allows for a duality result between the dual solution and the primal solution which we will then be used to estimate optimal plans and maps.

In the following, $\Omega_1 \subseteq \R^{n_1}$, $\Omega_2 \subseteq \R^{n_2}$ and $\Omega_1 \times \Omega_2 = \Omega \subseteq \R^{n_1 + n_2}$ will always be compact.

\todoinline{TODO:~figure out, what continuity $c$ is required to have}

\section{Regularized Optimal Transport}\label{RegOT}

There are multiple ways that regularization can take place, however we will focus on \textit{entropic regularization}. This form of regularization deals with positive integrable functions with finite entropy, i.e.~for $f \in \MF{\Omega}{\R}$
\[ E(f) := \int\limits_\Omega \vert f(z) \vert \log(\vert f(z) \vert)~\D[z] < \infty \]
with $0 \cdot \log(0) := 0$. From here we can define
\[ \LLL := \left\{ f \in \MF{\Omega}{\R} : \int\limits_\Omega \vert f(z) \vert \log^+(\vert f(z) \vert)~\D[z] < \infty \right\} \]
as a subset of all integrable functions with finite entropy, where we consider $\log^+(x) := \max \{ 0, \log (x) \}$.

The first conclusion that can be reached via~\cite{Cla2021}, Proposition~2.1, is the equivalence of $f \in \MF{\Omega}{\RZero}$ with $E(f) < \infty$ and $f \in \LLL$.

The second conclusion regards the structure of \LLL{}. When considering the \textit{Luxemburg norm}
\[ \Vert f \Vert_\Phi := \inf \left\{ C > 0 : \int\limits_\Omega \Phi\left( \frac{\vert f(z) \vert}{C} \right)~\D[z] \le 1 \right\} \]
for measurable functions $f$, we denote $\Orlicz{\Phi} := (\mathcal{F}, \Vert \cdot \Vert_\Phi)$, where $\mathcal{F} := \{ f \in \MF{\Omega}{\R} : \Vert f \Vert_\Phi < \infty \}$, the \textit{Orlicz space} for a certain function $\Phi$. Using $\Phi_{\log}(x) := x \log^+ (x)$, we have $\Orlicz{\Phi_{\log}} = \LLL$. In fact, by~\cite{Cla2021}, Theorem~2.5, \LLL{} is a Banach space with respect to its Luxemburg norm.

For a third conclusion we are interested in the dual space of \LLL{}. We first define
\[ \Phi_{\exp}(s) := \begin{cases}
	s & , 0 < s \le 1 \\
	\exp(s - 1) & , s > 1
\end{cases} \]
and then $\Lexp := \Orlicz{\Phi_{\exp}}$ in accordance to its corresponding Luxemburg norm $\Vert \cdot \Vert_{\Phi_{\exp}}$. As shown in~\cite{Cla2021}, Proposition~2.7, if $\Omega$ has finite Lebesgue measure (which it has by $\Omega \subseteq \R^{n_1 + n_2}$ being compact and thus bounded), then ${(\LLL)}^* = \Lexp$.

\begin{definition}[Regularized Problems; adapted from~\cite{Cla2021}, the beginning of Section~3 and Proposition~3.1]\label{RegProbs}
	We once again consider our starting measures $\mu \in \PM{\Omega_1}$ and $\nu \in \PM{\Omega_2}$. The \textbf{Regularized Problem} for some $\varepsilon > 0$ is defined as
	\[ \inf\limits_{\gamma \in \TP{\mu}{\nu}} \int\limits_\Omega c(x, y)~\Dx[x, y]{\gamma} + \varepsilon \int\limits_\Omega \gamma(x, y) \big( \log(\gamma(x, y)) - 1 \big)~\Dx[x, y]{}. \]
	Its \textbf{Dual Problem} for the same $\varepsilon > 0$ is defined as
	\[ \sup\limits_{\varphi \in \CF[b]{\Omega_1}, \psi \in \CF[b]{\Omega_2}} \int\limits_{\Omega_1} \varphi(x)~\Dx{\mu} + \int\limits_{\Omega_2} \psi(y)~\Dx[y]{\nu} - \varepsilon \int\limits_\Omega F_{\varepsilon}(x, y)~\Dx[x, y]{}, \]
	\[ \text{with } F_{\varepsilon}(x, y) := \exp\left( \frac{1}{\varepsilon} \big( \varphi(x) + \psi(y) - c(x, y) \big) \right). \]
	Just as in Definition~\ref{MongeProb},~\ref{KanProb} and~\ref{KanDual}, we will use the abbreviations (RP), $\inf \text{(RP)}$, (DR) and $\sup \text{(DR)}$ for the problems and their solutions respectively. Further, by~\cite{Cla2021}, Proposition~3.1, if $\sup \text{(DR)}$ is finite, (RP) admits a minimizer.
\end{definition}

With Theorem~\ref{KPAdmitPolishLSC} we already saw that an optimal transport plan exists for (KP). For (RP) however, we require more constraints on our marginal measures.

\begin{theorem}[Taken from~\cite{Cla2021}, Theorem~3.3]\label{RegProbAdmitLLL}
	The problem$\text{\normalfont\ (RP)}$ admits a minimizer $\gamma \in \TP{\mu}{\nu}$ if and only if $\mu \in \LLL[\Omega_1]$ and $\nu \in \LLL[\Omega_2]$. In this case, $\gamma \in \LLL$ and $\gamma$ is unique.
\end{theorem}

\begin{proof}
	Cf.~the proof of Theorem~3.3 in~\cite{Cla2021}.
\end{proof}

\todoinline{TODO:~\cite{Cla2021}, Theorem~4.6 and Theorem~4.7, then onto Equation~4.4 in Theorem~4.8}

\section{Optimal Transport Plan Estimation}\label{OTPlanEst}

\todoinline{TODO:~Stochastic estimation, starting from regularized OT~\cite{Seg2018}, Theorem 3.1 and Algorithm 1}

\section{Optimal Transport Map Estimation}\label{OTMapEst}

\todoinline{TODO:~Mapping estimation, starting with~\cite{Seg2018}, Definition 4.1}