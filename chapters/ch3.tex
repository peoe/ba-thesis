\chapter{Optimal Transport}

This section will be dealing with the problem of optimal transport and some results regarding the existence and uniqueness of such transports. There are two important paradigms in the theory of optimal transport. The original formulation is the so called Monge Problem. It was later relaxed by Kantorovich to a more general problem which is easier to handle.

\section{The Monge Problem}
To motivate the formulation of the Monge problem, we consider a mass modelled by a measure on $X$ to be moved to another mass, again modelled by a measure but on $Y$. One might for example imagine a pile of sand or a certain amount of particles being moved from one location to a different one. This transport will incur costs and the goal of optimal transport is to find a ``way of transporting'' the mass to its target whilst accumulating the smallest cost possible, while making sure that no mass is lost.

For this purpose we consider a \textit{cost function} $c(x, y)$, which will give us the cost of moving a particle from position $x$ to position $y$. Ideally this cost function will be continuous or semi-continuous.

To attain the Monge formulation, the transport is modelled via the \textit{push-forward measure} of our original measure. We will later see, that a more general approach to this transport is beneficial in solving for an optimal solution. For now however, we will constrain ourselves to the earlier approach.

\begin{definition}[Push-Forward]\label{PushForward}
	Given a measure $\mu \in \M{X}$ and a measurable map $T: X \rightarrow Y$, we define the \textbf{push-forward measure} of $\mu$ under $T$ as a measure on $Y$ by
	\[ \push{\mu}(y) := \mu(T^{-1} (y)). \]
\end{definition}

A very useful consequence of the push-forward is, that we can use it to state the conservation of mass throughout the transport. Imagine we have a starting measure $\mu$ on $X$ and a target measure $\nu$ on $Y$. In order to not lose any mass, it is necessary for
\[ \push{\mu} = \nu \]
to hold. Formally, this means that all elements of the underlying $\sigma$-algebra $\mathcal{Y}$ on $Y$ coincide under the measures $\nu$ and \push{\mu}, i.e.
\[ \forall A \in \mathcal{Y}: \push{\mu}(A) = \nu(A).\]

Another useful step to take is to norm the overall mass. This can be acheived by considering only probability measures on $X$ and $Y$.

We can now formulate the Monge Problem in terms of a cost function and the push-forward. This is a modern version of the provlem originally described by Gaspard Monge in 1781~\cite{Mon1781}.

\begin{definition}[Monge Problem]\label{MongeProb}
	Let $\mu \in \PM{X}, \nu \in \PM{Y}$, and $c: X \times Y \rightarrow \RZero \cup \infty$ be a cost function. Then we define the \textbf{Monge Problem} as
	\begin{equation}
		\inf \left\{ \int\limits_X c(x, T(x))~d\mu(x) : T \in \MF{X}{Y}, \push{\mu} = \nu \right\}.
	\end{equation}
	From here on, we will be using the abbreviation (MP) as the name of the Monge Problem, and $\inf \text{(MP)}$ will denote the attained infimum.
\end{definition}

\begin{definition}[Optimal Transport Map]\label{OTM}
	A function $T~\in~\MF{X, Y}$ is called an \textbf{optimal transport map}, if it satisfies
	\[ \int\limits_X c(x, T(x))~d \mu(x) = \inf \text{(MP)}. \]
\end{definition}

This problem in general is not easy to solve. For this, we can provide a few reasons:

Firstly, if one of the two measures were to be discrete, for example of the form $\mu = \delta_{x_0}, x_0 \in X$, and the other is not, (MP) would be infeasible. This can be seen, as $\forall y \in Y: T^{-1}(y) = x_0$ is a requirement for $\push{\mu} = \nu$. In order for $\mu$ to be transported to $\nu$ the transport would require to perform \textit{mass splitting}, which the transport map $T$ does not permit.

Secondly, as~\cite{San2015} pointed out, the constraint on $T$ in (MP) is not closed under weak convergence. For this we consider the sequence $T_n(x) := \sin(nx)$ on $[0, 2 \pi]$ for \NinN. As
\[ \int_0^{2 \pi} \sin^2(nx)~dx = \frac{4 \pi n - \sin(4\pi n)}{4n} = \pi < \infty, \]
we get ${(T_n)}_{\NinN} \in \ELL{2}[0, 2 \pi]$. Since for $1 = \frac{1}{p} + \frac{1}{q}$ we have $(\ELL{p})' = \ELL{q}$

\todoinline{Continue example cf.~\cite{San2015} just after formula 1.1 or Ex. 1}

Test citation:~\cite{Kan1942}
Test citation:~\cite{Pey2019}

\section{Kantorovich Relaxation}