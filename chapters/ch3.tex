\chapter{Optimal Transport}\label{OT}

This introductory part will follow\ \cite[Chapter~1]{San2015} closely in content as well as notation. It will contain the Monge and Kantorovich formulations of the optimal transport problem, as well as a dual problem for the Kantorovich problem. Afterwards, we will introduce regularized optimal transport and work towards mapping estimation in the second part.

\section{The Monge Problem}\label{TheMonProb}
%This section will deal with the problem of optimal transport and some results regarding the existence and uniqueness of such transports.
There are two important paradigms in the theory of optimal transport. The original formulation is the so-called Monge Problem. It was later relaxed by Kantorovich to a more general problem which is easier to handle. To motivate the modern formulation of the Monge problem, which was first described by\ \cite{Mon1781} in 1781, we consider a mass distribution modelled by a measure on $X$ to be moved to another mass distribution, again modelled by a measure, on $Y$. One may for example imagine a pile of sand or a number of particles being moved from one location to another. This transport will incur costs and the goal of optimal transport is to find a ``way of transporting'' the mass, which accumulates the smallest cost possible while making sure that no mass is lost.

For this purpose, we consider a non-negative \textit{cost function} $c(x, y)$, which will give us the cost of moving a particle from position $x$ to position $y$. Ideally this cost function will be continuous or semi-continuous, though later on we will see that we can obtain optimality and uniqueness in a simpler fashion by making use of more constraints on $c$.

To attain the Monge formulation, the transport is modelled via the \textit{push-forward measure}, or \textit{image measure}, of our original measure on $X$.

\begin{definition}[Push-Forward; adapted from\ {\cite[Section~9.1]{Bog2007}}]\label{PushForward}
	Given a measure $\mu \in \M{X}$ and a measurable map \map[T]{X}{Y}, we define the \textbf{push-forward measure} of $\mu$ under $T$ as a measure on $Y$ as
	\[ (\push{\mu})(y) := \mu\big( T^{-1} (y) \big). \]
\end{definition}

A very useful consequence of this definition is that we can use it to state the conservation of mass throughout the transport. Take a starting measure $\mu$ on $X$ and a target measure $\nu$ on $Y$. In order to not lose any mass, it is necessary for
\[ \push{\mu} = \nu \]
to hold. This means that all elements of the underlying $\sigma$-algebra $\mathcal{Y}$ on $Y$ coincide under the measures $\nu$ and \push{\mu}, i.e.
\[ \forall A \in \mathcal{Y}: (\push{\mu})(A) = \nu(A).\]
Another useful step to take is to normalize the overall mass. This can be achieved by considering only probability measures on $X$ and $Y$.

We can now state the Monge problem in its modern form:

\begin{definition}[Monge Problem; adapted from\ {\cite[Problem~1.1]{San2015}}]\label{MongeProb}
	Let $\mu \in \PM{X}, \nu \in \PM{Y}$ be our starting and target measures, and consider \map[c]{X\times{}Y}{\RZero{}\cup{}\{\infty\}}{} to be a cost function. Then we define the \textbf{Monge Problem} as
	\[ \inf \left\{ M(T) := \int\limits_X c\big( x, T(x) \big)~\Dx{\mu} : T \in \MF{X}{Y}, \push{\mu} = \nu \right\}. \]
	From here on, we will be using the abbreviation (MP) as the name of the Monge Problem, and $\inf \text{(MP)}$ to denote the attained infimum. A function $T~\in~\MF{X, Y}$ is called an \textbf{optimal transport map}, if it satisfies
	\[ \int\limits_X c\big( x, T(x) \big)~\Dx{\mu} = \inf \text{(MP)}. \]
\end{definition}

We can already provide a couple of results about the general existence of transport maps --- though these must not be optimal for (MP). The most prevalent restriction on our starting measures $\mu$ regards the existence of so-called \textit{atoms}.

\begin{definition}[Atoms; adapted from\ {\cite[Volume~1, Definition~1.12.7]{Bog2007}}]\label{Atoms}
	Let $\mu \in \M{X}$ be a measure and \X{} be a $\sigma$-algebra on $X$. A set $A \in \X$ is called an \textbf{atom} of $\mu$ if $\forall B \in \X, B \subseteq A: \text{either } \mu(B) = 0 \text{ or } \mu(B) = \mu(A)$ holds. If $\mu$ has no atoms, we say it is \textbf{atomless}.
\end{definition}

When there are no atoms for $\mu$ and the underlying measurable space is compact, we can first acheive existence of transport maps on the real line.

\begin{lemma}[Adapted from\ {\cite[Lemma~1.27]{San2015}}]\label{1DTransMapExist}
	Let $\mu, \nu \in \PM{\Omega}$ be two measures with $\mu$ being atomless, and $\Omega \subseteq \R$ being compact. Then, there exists a transport map \map[T]{\Omega}{\Omega} such that $\push{\mu} = \nu$.
\end{lemma}

\begin{proof}
	Cf.~the proof of\ \cite[Lemma~1.27]{San2015}.
\end{proof}

%\begin{lemma}[Adapted from~\cite{San2015}, Lemma 1.28]\label{BorelExist}
%	There exists an injective Borel-measurable map \map[\sigma_d]{\R^d}{\R} with its image $\text{im}(\sigma_d)$ being a Borel-subset of $\R$ and its inverse map being Borel-measurable as well.
%\end{lemma}

%\begin{proof}
%	Cf.~the proof of Lemma 1.28 in~\cite{San2015}.
%\end{proof}

By applying the injective map \map[\sigma_d]{\R^d}{\R} from\ {\cite[Lemma~1.29]{San2015}}, we can extend Lemma~\ref{1DTransMapExist} onto compact subsets of $\R^d$.

\begin{theorem}[Adapted from\ {\cite[Corollary~1.29]{San2015}}]\label{NDTransMapExist}
	Let $\mu, \nu \in \PM{\Omega}$ be two measures with $\mu$ being atomless, and $\Omega \subseteq \R^d$ being compact. Then, there exists at least a transport map \map[T]{\Omega}{\Omega} such that $\push{\mu} = \nu$.
\end{theorem}

\begin{proof}
	Cf.\ the proof of\ \cite[Corollary~1.29]{San2015}.
\end{proof}

%\begin{proof}
%	Using $\sigma_d$ from Lemma~\ref{BorelExist} we obtain the measures \push[(\sigma_d)]{\mu} and \push[(\sigma_d)]{\nu} on $\tilde{\Omega} \subseteq \R$ a compact subset of \R. By Lemma~\ref{1DTransMapExist} there now exists a transport map \map[\tilde{T}]{\tilde{\Omega}}{\tilde{\Omega}} which satisfies $\push[\tilde{T}]{(\push[(\sigma_d)]{\mu})} = \push[(\sigma_d)]{\nu}$. Applying the push-forward with the inverse $\sigma_d^{-1}$ we finally obtain
%	\[ \push{\mu} := \push[\sigma_d^{-1}]{(\push[\tilde{T}]{(\push[(\sigma_d)]{\mu})})} = \push[\sigma_d^{-1}]{(\push[(\sigma_d)]{\nu})} = \nu. \]
%\end{proof}

%\begin{remark}\label{TransMapExist}
%	It can be shown that for any atomless measure $\mu \in \PM{\R^d}$ (cf.\ \cite{Bog2007}, Volume 1, Definition 1.12.7) and any $\nu \in \PM{\R^d}$, there exists a transport map \map[T]{\R^d}{\R^d} with $\push{\mu} = \nu$. For a more detailed explanation, one may consider the argument made by~\cite{San2015} in Lemma~1.27 upto Corollary~1.29.
%\end{remark}

In general, (MP) is not easy to solve. In order to point out the difficulties we may face, we require the notion of \textit{weak convergence} of a measure.

%\begin{definition}[Dual Space; adapted from~\cite{Ryn2008}, Definition 4.28]\label{DuaSpa}
%	Let $(X, ||\cdot||)$ be a normed space over \K. We define the \textbf{dual space} of X as the set of all continuous linear functions \map{X}{\K}. Bestowed with the norm $||\varphi||_{X'} := \sup \{ |\varphi(x)| : ||x|| \le 1 \}$, the dual space is a normed space as well.
%\end{definition}

%\begin{definition}[Weak Convergence for Dual Spaces; adapted from~\cite{Ryn2008}, Definition 5.68]\label{WeakCon}
%	Let $(X, ||\cdot||)$ be a normed space, and $(X', ||\cdot||_{X'})$ its dual space. A sequence ${(x_n)}_{\NinN}$ is said to \textbf{converge weakly} to $x \in X$, if
%	\[ \forall \varphi \in X': \varphi(x_n) \xrightarrow[\Ninf]{} \varphi(x) \]
%	holds. In this case, we write \weak{x_n}{x} for \Ninf. The convergence of ${(\varphi(x_n))}_{\NinN}$ is to be understood in the norm $||\cdot||_{X'}$.
%\end{definition}

\begin{definition}[Weak Convergence of Measures; adapted from\ {\cite[Volume~2, Definition~8.1.1]{Bog2007}}]\label{WeakConMeas}
	Let $X$ be a measurable space and ${(\mu_n)}_{\NinN}$ be a sequence of measures which satisfies $\forall \NinN: \mu_n \in \M{X}$. This sequence is said to \textbf{converge weakly to a measure} $\mu \in \M{X}$, if for all $f \in \CF[b]{X}$ 
	\[ \lim\limits_{\NinN} \int\limits_X f(x)~\Dx{\mu_n} = \int\limits_X f(x)~\Dx{\mu} \]
	holds. In this case we will write \weak{\mu_n}{\mu} for \Ninf{}.
\end{definition}

\begin{remark}
	Technically, the space of radon measures is the dual of the space of continuous functions, see e.g.\ \cite[Section~4]{Kapl1957}. Following this duality, one can define weak-* convergence for locally finite non-negative measures in accordance with\ \cite[Definition~2.4]{Cami2008}. In Definition~\ref{WeakConMeas}, the observed measures are globally finite, meaning that it should instead refer to the \textit{weak-* convergence} of these measures. It is however not uncommon to see the usage of weak convergence in relevant literature. Due to this, we will continue to use this term, while at the same time acknowledging that it should not be used.
\end{remark}

We can now consider the following two points to illustrate the difficulties in dealing with (MP):

Firstly, if one of the two measures were to be discrete, for example of the form $\mu = \delta_{x_0}, x_0 \in X$, and the other is not, (MP) would turn infeasible. In order for $T$ to be a transport map for (MP), no \textit{mass splitting} must occur. With a semi-discrete transport problem, this is not possible.

Secondly, as~\cite{San2015} pointed out, the constraint on $T$ in (MP) is not closed under weak convergence (cf.\ Definition~\ref{WeakConMeas}). Consider the sequence $T_n(x) := \sin(nx)$ on $[0, 2 \pi]$ for \NinN. We get $(\push[T_n]{\mu})(y) = \mu\left( \frac{\arcsin(y)}{n} \right)$ and notice that for all $\varepsilon > 0$ there exists an \NinN{} such that the inner term on the right-hand side is less than $\varepsilon$ for all $y \in Y$. If \push[T_n]{\mu} is weakly convergent, it can thus only converge to $\nu \equiv 0$, i.e.\ for \NinN{} sufficiently large, $T_n$ no longer induces a probability measure via the push-forward.

Both of these problems can be avoided by considering a different way of formulating the transport constraint. We will later see that this relaxed description indeed includes (MP).

\section{The Kantorovich Relaxation}\label{KantRelax}
In 1942,\ \cite{Kan1942} gave a more general formulation for the optimal transport problem. Instead of formulating the transport via the push-forward under a transport map, Kantorovich used so-called \textit{transport plans}, where the original measures appeared as marginals of a joint measure. This, as~\cite{San2015} puts it,  allowed for a description of how many particles are moved from $x$ to $y$ instead of specifying the destination of $x$ under $T$.

\begin{definition}[Transport Plans; adapted from\ {\cite[Problem~1.2]{San2015}}]\label{TransPlans}
	For two measures $\mu \in \PM{X}, \nu \in \PM{Y}$ we define the set of all \textbf{transport plans} from $\mu$ to $\nu$ as
	\[ \TP{\mu}{\nu} := \Big\{ \gamma \in \PM{X \times Y} : \push[(\pi_X)]{\gamma} = \mu, \push[(\pi_Y)]{\gamma} = \nu \Big\}, \]
	where $\pi_X$ and $\pi_Y$ are the canonical projections onto $X$ and $Y$, i.e.
	\[ \pi_X(x, y) := x, \text{ and } \pi_Y(x, y) := y. \]
\end{definition}

Using this terminology we can now describe the Kantorovich Problem as a more general problem when compared to (MP).

\begin{definition}[Kantorovich Problem; adapted from\ {\cite[Problem~1.2]{San2015}}]\label{KanProb}
	Let $\mu \in \PM{X}, \nu \in \PM{Y}$ be our starting measures, and consider the function $\map[c]{X \times Y}{\RZero \cup \{\infty\}}$ to be a cost function. Then we call the following optimization problem the \textbf{Kantorovich Problem}:
	\[ \inf \left\{ K(\gamma) := \int\limits_{X \times Y} c(x, y)~\Dx[x, y]{\gamma} : \gamma \in \TP{\mu}{\nu} \right\}. \]
	In analogy to Definition~\ref{MongeProb} we will be using the abbreviation (KP) to describe the problem itself and denominate the attained infimum as $\inf \text{(KP)}$. We shall further define \textbf{optimal transport plans} as those specific transport plans $\gamma \in \TP{\mu}{\nu}$ that satisfy
	\[ \int\limits_{X \times Y} c(x, y)~\Dx[x, y]{\gamma} = \inf \text{(KP)}. \]
\end{definition}

One direct benefit of this formulation is that we are no longer bound by the prohibition of mass splitting like in (MP). We can however recover the (MP) formulation via (KP) if $\gamma = \push[(id, T)]{\mu}$ is an optimal transport plan for $T \in \MF{X}{Y}$, in which case $T$ would once again be an optimal transport map. If given two measures $\mu \in \M{X}, \nu \in \M{Y}$ and a transport map $T \in \MF{X}{Y}$ with $\push{\mu} = \nu$, we can always induce a transport plan from $T$ by defining
\begin{equation}\label{IndPlan}
	\gamma_T := \push[(id, T)]{\mu}.
\end{equation}

We are now interested in the relation between (KP) and (MP). First we define
\[ J(\gamma) := 
	\begin{cases}
		K(\gamma) = M(T), & \gamma = \gamma_T \\
		\infty, & \text{otherwise}
	\end{cases}. \]
The functional $J$ immediately allows us to consider (MP) on the same set of admissible objects as (KP), that is all transport plans induced by a transport map. As Kantorovich replaced $J$ with $K$ the question to be asked changes to: does $\inf K = \inf J$ hold?

To answer this question, it can be shown that the set of plans induced by a map is dense in \TP{\mu}{\nu}, whenever certain conditions are met. This will then allow the conclusion that, indeed, both infima coincide. For these results, we will always require compactness for a subset $\Omega \subseteq \R^d$, however~\cite{San2015} points out that this is just for simplicity and more general statements can be made.

%\begin{lemma}[Taken from~\cite{San2015}, Lemma 1.31]\label{CompWeakConv}
%	Consider on a compact metric space $X$, endowed with a probability $\rho \in \PM{X}$, a sequence of partitions $G_n$, each $G_n$ being a family of disjoint subsets $C_{i, n}$ such that \[ \bigcup\limits_{i \in I_n} C_{i, n} = X \] for every \NinN. Suppose that $\max\limits_{i \in I_n} \text{\normalfont\ diam}(C_{i, n})$ tends to $0$ as \Ninf{} and consider a sequence of probability measures $\rho_n$ on $X$ such that, for every \NinN{} and $i \in I_n$, we have $\rho_n(C_{i, n}) = \rho(C_{i, n})$. Then \weak{\rho_n}{\rho} for \Ninf.
	
%	Here, the diameter of a set $A \subseteq X$ is to be understood as
%	\[ \text{\normalfont\ diam}(A) := \sup \{ d(x, y) : x, y \in A \}. \]
%\end{lemma}

%\begin{proof}
%	Cf.~the proof of Lemma 1.31 in~\cite{San2015}.
%\end{proof}

\begin{theorem}[Plans Induced by Maps are Dense; taken from\ {\cite[Theorem~1.32]{San2015}}]\label{IndPlansDense}
	Let $\Omega \subseteq \R^d$ be a compact subset, and $\mu, \nu \in \PM{\Omega}$ be two probability measures. If $\mu$ is atomless, then the set of all transport plans $\gamma_T$ induced by a transport map $T$ as defined in Equation~\ref{IndPlan} is dense in the set of all transport plans \TP{\mu}{\nu}.
\end{theorem}

\begin{proof}
	Cf.~the proof of\ \cite[Theorem~1.32]{San2015}.
\end{proof}

Before we can prove strong duality for continuous cost functions and atomless $\mu$, we need to define the term \textit{lower semi-continuous}. This definition will then be used to consider the \textit{relaxation} of the functional $J$ in the following proof.

\begin{definition}[Lower Semi-Continuous Functions; adapted from\ {\cite[Definition~5.1.2]{Kes2009}}]\label{lsc}
	Let $X$ be an arbitrary topological space. Then a function \map[f]{X}{\R} is called \textbf{lower semi-continuous} (l.s.c.), if the set
	\[ \big\{ x \in X : f(x) \le \alpha \big\} \]
	is closed in $X$ for all $\alpha \in \R$. In particular, if \map[f]{X}{\R} is continuous, it is also l.s.c.
\end{definition}

\begin{theorem}[Infima of J and K coincide; adapted from\ {\cite[Theorem~1.33]{San2015}}]\label{InfCoincide}
	Let $\,\Omega \subseteq \R^d$ be a compact subset, $\mu, \nu \in \PM{\Omega}$ be two measures on $\Omega$ with $\mu$ being atomless, and $\map[c]{\Omega \times \Omega}{\RZero \cup \{ \infty \}}$ be a continuous cost function. Then $\inf J = \inf K$ holds, with the infimum being taken over \TP{\mu}{\nu}.
\end{theorem}

\begin{proof}
	This proof was adapted from\ \cite[the proof of Theorem~1.33 and the memo on relaxations, Box~1.10]{San2015}.

	We consider the relaxation $\overline{J}$ of $J$ defined by
	\[ \forall \gamma \in \TP{\mu}{\nu}: \overline{J}(\gamma) := \inf \left\{ \liminf\limits_{\Ninf} J(\gamma) : \weak{\gamma_n}{\gamma}, \Ninf \right\}. \]
	Since $J \ge \overline{J}$ per definition implies $\inf J \ge \inf \overline{J}$, and as $J \ge \inf J$ results in $\overline{J} \ge \inf J$ because $\inf J$ is constant and thus l.s.c., we get that $\inf J = \inf \overline{J}$. As $K$ is also l.s.c.\ with $K \le J$ we obtain $\inf K \le \inf J$.

	To show $\inf K = \inf J$ we need to find a sequence of transport maps ${(T_n)}_{\NinN}, T_n \in \MF{\R^d}{\R^d}$ with $\push[(T_n)]{\mu} = \nu$ for every $\gamma \in \TP{\mu}{\nu}$, such that $\gamma_{T_n} \rightharpoonup \gamma$ and $J(\gamma_{T_n}) \rightarrow K(\gamma)$ for \Ninf. Because $K$ is continuous and for $\gamma = \gamma_{T_n}$ it holds that $K(\gamma) = J(\gamma)$, it is already sufficient for a sequence ${(T_n)}_{\NinN}$ to exist with \weak{\gamma_{T_n}}{\gamma} for \Ninf.

	Such a sequence exists, since the set of transport plans induced by a transport map is dense in \TP{\mu}{\nu} by Theorem~\ref{IndPlansDense}. With this minimizing sequence we finally have $\inf K = \inf \overline{J} = \inf J$.
\end{proof}

So far we have avoided the question, if (KP) even admits a solution. We will now give a basic result for continuous cost functions. One of the essential ideas here is the continuity of the functional $K$. For this, an explicit notion of convergence is needed. We will be referring to the continuity of $K$ in terms of the weak convergence of measures (cf.\ Definition~\ref{WeakConMeas}).

\begin{lemma}[Taken from\ {\cite[Theorem~1.4]{San2015}}]\label{KPAdmitCompCont}
	Let $X$ and $Y$ be compact metric spaces, $\mu \in \PM{X}$, $\nu \in \PM{Y}$ be probability measures, and consider \map[c]{X \times{} Y}{\R} to be a continuous function. Then$\text{\normalfont\ (KP)\,}$admits a solution.
\end{lemma}

\begin{proof}
	This proof was adapted from the proof of\ \cite[Theorem~1.4]{San2015}. It is sufficient to show that $\TP{\mu}{\nu}$ is compact and $\gamma \mapsto K(\gamma)$ is continuous, as any continuous function attains its minimum over a compact set.

	To show the continuity, let ${(\gamma_n)}_{\NinN}$ be a sequence of probability measures over $X \times Y$ converging weakly to $\gamma \in \PM{X \times Y}$. As $c$ is continuous and especially measurable, we have
	\[ \lim\limits_{\NinN} K(\gamma_n) = \lim\limits_{\NinN} \int\limits_{X \times Y} c~\D[\gamma_n] = \int\limits_{X \times Y} c~\D[\gamma] = K(\gamma). \] 
	
	To show the compactness, let ${(\gamma_n)}_{\NinN}$ be a sequence with $\gamma_n \in \TP{\mu}{\nu}$ for every \NinN. Let $f \in \CF[b]{X \times Y}$. We define $\hat{f}_n := \int f~\D[\gamma_n]$. With $\lvert \int f~\D[\gamma_n] \rvert \le \int \lvert f \rvert~\D[\gamma_n] \le \sup\limits_{X \times Y} |f| < \infty$, because $f$ is bounded and continuous on a compact set, we see that the sequence ${(\hat{f}_n)}_{\NinN}$ is bounded. Hence, for every $f \in \CF[b]{X \times Y}$ there exists a convergent subsequence ${(\hat{f}_{n_k})}_{\NinN[k]}$.
\end{proof}

\begin{remark}
	Compactness and sequential compactness are the same on metric spaces. To be able to view the set of transport plans as a compact set in a metric space, one may consider the linear space of all bounded and signed measures over $X \times Y$ with the norm $\Vert \pi \Vert := |\pi|(X)$. The triangle inequality follows from the Hahn decomposition $X \times Y = A \dot{\cup} B$, such that $(\sigma + \tau)(A) \le 0$, and $(\sigma + \tau)(B) \ge 0$ (cf.\ \cite[Volume~1, Theorem~3.1.1 and Corollary~3.1.2]{Bog2007}), using
	\[ \Vert \sigma + \tau \Vert = \vert \sigma + \tau \vert (X) = (\sigma + \tau)\big( (X \times Y) \cap A \big) - (\sigma + \tau)\big( (X \times Y) \cap B \big) \]
	\[ = (\sigma + \tau)(A) - (\sigma + \tau)(B) \le \vert \sigma \vert (A) + \vert \sigma \vert (B) + \vert \tau \vert (A) + \vert \tau \vert (B) = \Vert \sigma \Vert + \Vert \tau \Vert. \]
	As our transport plans are probability measures on $X \times Y$, they are bounded and countably additive. We can thus assume \TP{\mu}{\nu} to be a subset of the metric space of signed measures.
\end{remark}

In most cases, our cost functions will be sufficiently continuous in order to apply Lemma~\ref{KPAdmitCompCont} and guarantee the existence of a solution to (KP). However, it is still possible to generalize to l.s.c.\ cost functions on compact subspaces, and even l.s.c.\ cost functions on complete and separable metric spaces (i.e.\ \textit{Polish spaces}).

\begin{lemma}[Taken from\ {\cite[Theorem~1.5]{San2015}}]\label{KPAdmitCompLSCBound}
	Let $X$ and $Y$ be any compact metric spaces, $\mu \in \PM{X}$, $\nu \in \PM{Y}$ be probability measures, and finally $\map[c]{X \times Y}{\R \cup \{ \infty \}}$ be an l.s.c.\ function bounded from below. Then$\text{\normalfont\ (KP)\,}$admits a solution.
\end{lemma}

\begin{proof}
	Cf.~the proof of\ \cite[Theorem~1.5]{San2015}.
\end{proof}

\begin{theorem}[Adapted from\ {\cite[Theorem~1.7]{San2015}}]\label{KPAdmitPolishLSC}
	Let $X$ and $Y$ be complete and separable metric spaces, $\mu \in \PM{X}$, $\nu \in \PM{Y}$ be probability measures, and $\map[c]{X \times Y}{\RZero \cup \{ \infty \}}$ be l.s.c. Then$\text{\normalfont\ (KP)\,}$admits a solution.
\end{theorem}

\begin{proof}
	This proof was adapted from the proof of\ \cite[Theorem~1.7]{San2015}. Our objective is to show that any sequence in \TP{\mu}{\nu} is tight and apply the Theorem of Prokhorov (\ref{Prok}), as we can no longer rely on the finite maximum of continuous functions over our (in Lemma~\ref{KPAdmitCompLSCBound} still compact) space $X \times Y$.

	To this end, let $\varepsilon > 0$. As a sequence made up of a single transport plan $\gamma$ is necessarily tight, we can find two compact sets $K_X \subseteq X, K_Y \subseteq Y$, such that $\mu(X \setminus K_X) < \frac{\varepsilon}{2}$ and $\nu(Y \setminus K_Y) < \frac{\varepsilon}{2}$ hold true. Hence, the set $K_X \times K_Y \subseteq X \times Y$ is compact too, resulting in
	\[ \forall \gamma_n \in \TP{\mu}{\nu}: \gamma_n\big( (X \times Y) \setminus (K_X \times K_Y) \big) \]
	\[ \le \gamma_n\big( (X \setminus K_X) \times Y \big) + \gamma_n\big( X \times (Y \setminus K_Y) \big) = \mu(X \setminus K_X) + \nu(Y \setminus K_Y) < \varepsilon. \]
	With Prokhorov (\ref{Prok}) we get \weak{\gamma_n}{\gamma} for \Ninf. As our transport plan $\gamma$ was chosen arbitrarily, this property holds for all transport plans and thus all sequences of transport plans. Finally, the claim can be seen using the compactness of \TP{\mu}{\nu} and the continuity of $K$ with regard to the weak convergence of measures (cf.\ Definition~\ref{WeakConMeas}).
\end{proof}

It is beneficial to investigate the dual problem of (KP), because this dual problem will enable us to consider a regularized dual optimal problem in the second part of the thesis. This regularized dual will ultimately aid in the estimation of optimal transport plans and consequently in the estimation of maps as well.

\section{Kantorovich Duality}\label{KanDual}

As~\cite{San2015}, Section~1.2, points out, the problem (KP) is a linear optimization problem with convex constraints. Hence, it is plausible to investigate the \textit{dual problem}. Beforehand, we introduce the notion of the \textit{support} of a function.

\begin{definition}[Support; adapted from\ {\cite[Definition~1.14]{San2015}}]\label{Supp}
	Let $X$ be a separable metric space, and $\mu \in \M{X}$ be a measure on $X$. We define the \textbf{support} of $\mu$ as $\supp{\mu} := \bigcap \big\{ A \subseteq X : A \text{ closed and } \mu(X \setminus A) = 0 \big\}$.
\end{definition}

To get a dual problem, it is useful to restate the constraint ``$\gamma \in \TP{\mu}{\nu}$'' in terms of continuous bounded functions. We notice that for a general measure $\gamma \in \M{X \times Y}$ the term
\[ \sup\limits_{\varphi \in \CF[b]{X}, \psi \in \CF[b]{Y}} \int\limits_X \varphi(x)~\Dx{\mu} + \int\limits_Y \psi(y)~\Dx[y]{\nu} - \int\limits_{X \times Y} \varphi(x) + \psi(y)~\Dx[x, y]{\gamma} \]
equals $0$ if $\gamma \in \TP{\mu}{\nu}$, and $\infty$ if otherwise. This effectively constitutes a restatement of our constraint if we add it to the original formulation of (KP): when the constraint is fulfilled, nothing has changed, and when $\gamma$ is not a transport plan, (KP) stays infeasible. For ease of notation, we will be using $(\varphi \oplus \psi)(x, y) := \varphi(x) + \psi(y)$. Exchanging the $\inf$ and the $\sup$ in the modified (KP), we obtain
\[ \sup\limits_{\varphi \in \CF[b]{X}, \psi \in \CF[b]{Y}} \int\limits_{X} \varphi~\D + \int\limits_{Y} \psi~\D[\nu] + \inf\limits_{\gamma \in \M[+]{X \times Y}} \int\limits_{X \times Y} c - \varphi \oplus \psi~\D[\gamma]. \]
We moreover want to restate the above infimum in $\gamma$ as a constraint on $\varphi$ and $\psi$. We get
\[ \inf\limits_{\gamma \in \M[+]{X \times Y}} \int\limits_{X \times Y} c - \varphi \oplus \psi~\D[\gamma] = 
\begin{cases}
	0, & \varphi \oplus \psi \le c \text{ on } X \times Y \\
	- \infty, & \text{else}
\end{cases}. \]
This equation holds,\ \cite{San2015} argues, because if $\varphi \oplus \psi > c$ somewhere in $X \times Y$, we can choose a measure $\gamma$ supported on that region with a mass tending to $\infty$. Such a $\gamma$ would leave the expression unbounded from below, i.e.\ the problem infeasible.

This argument allows us to state the dual problem to the Kantorovich relaxation of (MP). With this problem, we will then investigate the duality between (KP) and its dual.

\begin{definition}[Kantorovich Dual Problem; adapted from\ {\cite[Problem~1.9]{San2015}}]\label{DualProb}
	Let $\mu \in \PM{X}, \nu \in \PM{Y}$ be our starting measures, and consider $\map[c]{X \times Y}{\RZero{} \cup{} \{ \infty \}}$ to be a cost function. Then we call the following optimization problem the \textbf{Kantorovich Dual Problem}:
	\[ \sup \left\{ \int\limits_{X} \varphi~\D + \int\limits_{Y} \psi~\D[\nu] : \varphi \in \CF[b]{X}, \psi \in \CF[b]{Y}, \varphi \oplus \psi \le c \right\}. \]
	In analogy to Definitions~\ref{MongeProb}~and~\ref{KanProb} we will be using the abbreviation (DP) to describe the problem itself and $\sup \text{(DP)}$ to denominate its supremum.
\end{definition}

With the formulation of (DP) and the previous characterization of the constraint ``$\gamma \in \TP{\mu}{\nu}$'', it is now possible to reach a statement about strong duality. For all admissible $\gamma \in \TP{\mu}{\nu}$ (i.e.~the constraint from (KP)) and $\varphi \in \CF[b]{X}, \psi \in \CF[b]{Y}$ with $\varphi \oplus \psi \le c$ (i.e.~the constraint from (DP)) we have
\[ \sup\limits_{\varphi \in \CF[b]{X}, \psi \in \CF[b]{Y}} \int\limits_{X} \varphi(x)~\Dx{\mu} + \int\limits_{Y} \psi(y)~\Dx[y]{\nu} \]
\[ = \sup\limits_{\varphi \in \CF[b]{X}, \psi \in \CF[b]{Y}} \int\limits_{X \times Y} (\varphi \oplus \psi)(x, y)~\Dx[x, y]{\gamma} \le \int\limits_{X \times Y} c(x, y)~\Dx[x, y]{\gamma}. \]
As the left-hand side of this equality is constant, we can further consider the infimum over all $\gamma \in \TP{\mu}{\nu}$ and obtain $\sup \text{(DP)} \le \inf \text{(KP)}$.

So far, we do not know if the supremum on the left-hand side does exist. To handle this problem, we will be further transforming (DP) using so-called \textit{$c$-transforms}, or \textit{$c$-conjugate functions}. With these transformations it will then be possible to formulate (DP) as an optimization problem over just one dual variable.

\begin{definition}[$c$-Transform, $\bar{c}$-Transform, $c$-Concavity, $\bar{c}$-Concavity; taken from\ {\cite[Definition~1.10]{San2015}}]\label{cTrafo}
	Given \map[\chi]{X}{\Rbar} and \map[\zeta]{Y}{\Rbar} two functions, as well as $\map[c]{X \times Y}{\RZero{} \cup{} \{ \infty \}}$ a cost function, we define the \textbf{$\text{\textbf{\textit{c}}}$-transform} of $\chi$ by
	\[ \map[\chi^c]{Y}{\Rbar},~y \mapsto \inf\limits_{x \in X} c(x, y) - \chi(x) \]
	and the \textbf{$\bar{\text{\textbf{\textit{c}}}}$-transform} of $\zeta$ by
	\[ \map[\zeta^{\bar{c}}]{X}{\Rbar},~x \mapsto \inf\limits_{y \in Y} c(x, y) - \zeta(y). \]
	We further define the notion for a function \map[\psi]{Y}{\Rbar} to be \textbf{$\bar{\text{\textbf{\textit{c}}}}$-concave} if there exists a function \map[\chi]{Y}{\Rbar}, such that $\psi = \chi^c$. The set of all $\bar{c}$-concave functions over $Y$ will be denoted by \CBConc{Y}. Analogously, a function \map[\varphi]{X}{\Rbar} is said to be \textbf{$\text{\textbf{\textit{c}}}$-concave} if there exists a function \map[\zeta]{X}{\Rbar}, such that $\varphi = \zeta^{\bar{c}}$, and the set of all such $c$-concave functions over $X$ will accordingly be described by \CConc{X}.
\end{definition}

\begin{theorem}[Adapted from\ {\cite[Proposition~1.11]{San2015}}]\label{DPSolExist}
	Let $X$ and $Y$ be compact metric spaces and \map[c]{X \times{} Y}{\RZero} be a continuous cost function. Then there exists a solution $(\varphi,\psi) \in \CF[b]{X} \times \CF[b]{Y}$ of\text{\normalfont\ (DP)\,}with the form $\varphi \in \CConc{X}, \psi \in \CBConc{Y}$ with $\psi = \varphi^c$. In particular, we can restate
	\[ \sup\!\text{\normalfont\,(DP)} = \max\limits_{\varphi \in \CConc{X}} \int\limits_X \varphi~\D + \int\limits_Y \varphi^c~\D[\nu]. \]
\end{theorem}

\begin{proof}
	This proof was adapted from the proof of\ \cite[Proposition~1.11]{San2015}.

	We consider a maximizing sequence ${(\varphi_n, \psi_n)}_{\NinN}$, with $\varphi_n \oplus \psi_n \le c$ and $(\varphi_n, \psi_n) \in \CF[b]{X} \times \CF[b]{Y}$ for all \NinN{}, that is a pair of admissible functions for (DP). By this assumption we have $\forall x \in X, y \in Y$:
	\[ \psi_n(y) \le c(x, y) - \varphi_n(x), \]
	and note that $\int\limits_Y \psi_n~\D[\nu] \le \int\limits_Y \inf\limits_{x \in X} c(x, y) - \varphi_n(x)~\Dx[y]{\nu} = \int\limits_Y \varphi_n^c~\D[\nu]$. Thus it will suffice to consider a maximizing sequence ${(\varphi_n, \varphi_n^c)}_{\NinN}$. The same argument applied to $\varphi_n$ instead of $\psi_n$ results in a sequence ${(\psi^{\bar{c}}, \psi_n)}_{\NinN}$. This already means that for an arbitrary pair $(\varphi, \psi)$ to be a solution, it is necessary that $\varphi^c = \psi$ and $\psi^{\bar{c}} = \varphi$ hold, i.e.~that they must be $\bar{c}$-concave and $c$-concave respectively, since the integrals otherwise could be further enlarged by considering the $c$- and $\bar{c}$-transforms of said solution.

	From the proof of\ \cite[Proposition~1.34]{San2015}, we have $\varphi^{c\bar{c}} = \varphi, \psi^{\bar{c}c} = \psi$, as
	\[ \forall x \in X: \eta^{c\bar{c}}(x) = \inf\limits_{y \in Y} c(x, y) - \eta^c(y) = \inf\limits_{y \in Y} c(x, y) - \inf\limits_{z \in X} (c(z, y) - \eta(z)) \]
	\[ \ge \inf\limits_{y \in Y} c(x, y) - c(x, y) + \eta(x) = \eta(x), \]
	and for $\zeta^{\bar{c}c} \ge \zeta$ analogously. For $\eta$ $c$-concave we hence have
	\[ \eta^c = \chi^{\bar{c}c} \ge \chi \Rightarrow \eta^{c\bar{c}}(x) = \inf\limits_{y \in Y} c(x, y) - \eta^c(y) \le \chi^{\bar{c}}(x) = \eta(x) \Rightarrow \eta^{c\bar{c}} = \eta, \]
	where \map[\chi]{Y}{\Rbar} with $\eta = \chi^{\bar{c}}$. As such, any further iteration of these transformations cannot increase the dual value of sequence indefinitely.

	We now consider a sequence of $c$- or $\bar{c}$-transforms of our original sequence and once again label it as ${(\varphi_n, \psi_n)}_{\NinN}$. To be able to show convergence of this sequence, we want to apply the Theorem of Arzela-Ascoli (\ref{Arz-Asc}). To do so, we still need to show the equiboundedness and equicontinuity of our sequence.
	
	Let us assume that the sequence ${(\varphi_n, \psi_n)}_{\NinN}$ is not equibounded. In this case we have $\forall C > 0~\exists (x, y) \in X \times Y, \NinN[N]$:
	\[ |(\varphi_N, \psi_N)(x, y)| > C. \]
	Since $\varphi_n$ and $\psi_n$ are both bounded, we can assume that
	\[ \forall \NinN~\exists C_n > 0~\forall (x, y) \in X \times Y: |(\varphi_n, \psi_n)(x, y)| < C_n, \]
	and because ${(\varphi_n, \psi_n)}_{\NinN}$ is increasing, we can assume that the sequence ${(C_n)}_{\NinN}$ is increasing too. This means that for the specific \NinN[N] from the assumption we get $C_N > \max \{C_n : n < N \}$, thus ${(c_n)}_{\NinN}$ is strictly increasing. If ${(C_n)}_{\NinN}$ were to converge to a constant $\tilde{C} > 0$ for \Ninf{}, the assumption is contradicted for all $C > \tilde{C}$, as there could never exist a suitable \NinN. If ${(C_n)}_{\NinN}$ does not converge, there needs to exist a pair $(\hat{x}, \hat{y}) \in X \times Y$ with $|(\varphi_n, \psi_n)(\hat{x}, \hat{y})| > C_{n - 1}$. This too would lead to a contradiction, because $C_n \rightarrow \infty$ for \Ninf{}, and hence for a certain $\NinN[\tilde{N}]~\forall n \ge \tilde{N}: \varphi_n \oplus \psi_n > c$, as $c$ is continuous on a compact space and must thus attain a finite maximum. In this case, our sequence would no longer be admissible to (DP).
	
	Let us now assume, that the sequence ${(\varphi_n, \psi_n)}_{\NinN}$ is not equicontinuous, i.e.\ there exists an $\varepsilon > 0$, such that $\forall \delta > 0~\exists \NinN, (x, y), (\hat{x}, \hat{y}) \in X \times Y$, $\big\vert\big( d_X(x, \hat{x}), d_Y(y, \hat{y}) \big)\big\vert< \delta: |(\varphi_n, \psi_n)(x, y) - (\varphi_n, \psi_n)(\hat{x}, \hat{y})| \ge \varepsilon$. Because of this assumption however, we have already contradicted the uniform continuity of all $(\varphi_n, \psi_n)$ on the compact space $X \times Y$.
	
	We can now apply Arzela-Ascoli (\ref{Arz-Asc}) in order to obtain a subsequence ${(\varphi_{n_k}, \psi_{n_k})}_{\NinN[k]}$, which uniformly converges to a continuous and furthermore bounded function \map[(\varphi, \psi)]{X \times{} Y}{\R}. This subsequence is once again equibounded and convergent, such that we can further apply the Theorem of Lebesgue (\ref{Leb-DomCon}) to obtain
	\[ \lim\limits_{\NinN[k]} \int\limits_X \varphi_{n_k}~\D + \lim\limits_{\NinN[k]} \int\limits_Y \psi_{n_k}~\D[\nu] = \int\limits_X \varphi~\D + \int\limits_Y \psi~\D[\nu]. \]
	To show the admissibility of $(\varphi, \psi)$, we note that
	\[ \varphi_{n_k} \oplus \psi_{n_k} \le c \Rightarrow \varphi \oplus \psi \le c. \]
	Thus, $(\varphi, \psi)$ is a solution to (DP) satisfying the desired properties.
\end{proof}

So far we have seen that both (KP) and (DP) admit solutions, but up to now the relation between these solutions has only been $\sup \text{(DP)} \le \inf \text{(KP)}$. Going forward, we want to show $\inf \text{(KP)} \le \sup \text{(DP)}$. One way to do so, is the \textit{$c$-cyclical monotonicity} of the underlying set $\Omega \subseteq X \times Y$.

\begin{definition}[$c$-Cyclical Monotonicity; adapted from\ {\cite[Definition~1.36]{San2015}}]\label{c-CM}
	For any given function $\map[c]{X \times{} Y}{\R \cup \{ \infty \}}$, we say that a set $\Omega \subseteq X \times Y$ is $c$-cyclically monotone ($c$-CM), if for every \NinN[k], every permutation $\sigma \in S_k$ and every finite set of points $(x_1, y_1), \dots, (x_k, y_k) \in \Omega$ we have
	\[ \sum\limits_{i = 1}^k c(x_i, y_i) \le \sum\limits_{i = 1}^k c(x_i, y_{\sigma(i)}). \]
\end{definition}

For such $c$-cyclically monotone $\Omega \subseteq X \times Y$, we can achieve the following result for $c$-concave functions and $\Omega$.

\begin{lemma}[Adapted from\ {\cite[Theorem~1.37]{San2015}}]\label{c-CM-c-ConcExist}
	Let $\Omega \subseteq X \times Y$ be a $c$-CM set with $\Omega \neq \emptyset$, and \map[c]{X \times{} Y}{\R} be an arbitrary function. Then there exists a $c$-concave function $\map[\varphi]{X}{\R \cup \{ - \infty \}}$, $\varphi \not\equiv -\infty$, such that
	\[ \Omega \subseteq \big\{ (x, y) \in X \times Y : \varphi(x) + \varphi^c(y) = c(x, y) \big\}. \]
\end{lemma}

\begin{proof}
	Cf.~the proof of\ \cite[Theorem~1.37]{San2015}.
\end{proof}

In particular, it can be shown that the support of all optimal transport plans for (KP) are $c$-cyclically monotone.

\begin{lemma}[Adapted from\ {\cite[Theorem~1.38]{San2015}}]\label{OTPlanc-CM}
	Let $\gamma$ be an optimal transport plan for$\text{\normalfont\ (KP)\,}$with $\map[c]{X \times Y}{\RZero \cup \{ \infty \}}$ a continuous cost function. Then $\text{\normalfont{}supp}(\gamma)$ is a $c$-CM set.
\end{lemma}

\begin{proof}
	Cf.~the proof of\ \cite[Theorem~1.38]{San2015}.
\end{proof}

In combination with the concepts of $c$- and $\bar{c}$-concavity, Lemma~\ref{OTPlanc-CM} then results in strong duality between (KP) and (DP) for complete and separable metric spaces, and uniformly continuous and bounded cost functions.

\begin{theorem}[Adapted from\ {\cite[Theorem~1.39]{San2015}}]\label{BoundedPDRel}
	Let $X$ and $Y$ be complete and separable metric spaces, and \map[c]{X \times{} Y}{\R} be uniformly continuous and bounded. Then the problem$\text{\normalfont\ (DP)\,}$admits a solution $(\varphi, \varphi^c)$ and strong duality holds, i.e.\ $\sup \!\text{\normalfont\,(DP)} = \inf \!\text{\normalfont\,(KP)}$.
\end{theorem}

\begin{proof}
	This proof was adapted from the proof of\ \cite[Theorem~1.39]{San2015}.

	We will first consider (KP) in order to find an admissible pair for (DP). As $c$ is continuous and especially l.s.c., we can apply Theorem~\ref{KPAdmitPolishLSC} to obtain a solution $\gamma$ of (KP). By Lemma~\ref{OTPlanc-CM} the set $\Omega := \text{supp}(\gamma)$ is $c$-CM\@. Applying Lemma~\ref{c-CM-c-ConcExist} we can find a $c$-concave and thus continuous function $\varphi$, such that $\Omega \subseteq \big\{ (x, y) \in X \times Y : \varphi(x) + \varphi^c(y) = c(x, y) \big\}$. Both $\varphi$ and $\varphi^c$ must further be bounded: if we assume $\varphi$ and $\varphi^c$ to be unbounded, i.e.~for all $C \in \R: \sup \vert \varphi \oplus \varphi^c \vert > C$, our $c$ would also be unbounded.
	
	Because $(\varphi, \varphi^c)$ is an admissible pair for (DP), we have
	\[ \int\limits_X \varphi~\D + \int\limits_Y \varphi^c~\D[\nu] \overset{(*)}{=} \int\limits_{\Omega} \varphi \oplus \varphi^c~\D[\gamma]  = \int\limits_{\Omega} c~\D[\gamma] = \int\limits_{X \times Y} c~\D[\gamma]. \]
	Here, the equality $(*)$ holds because $\gamma$ is concentrated on $\Omega$ and $\varphi \oplus \varphi^c = c$ on $\Omega$.

	As $\gamma$ is an optimal transport plan, we get $\sup \text{(DP)} \ge \inf \text{(KP)}$. When this inequality is combined with our previously obtained $\sup \text{(DP)} \le \inf \text{(KP)}$ we get the claimed strong duality $\sup \text{(DP)} = \inf \text{(KP)}$.
\end{proof}

%The problem we now face, is that our $c$ can not always be assumed to be uniformly continuous or bound from above (the bound from below is always satisfied, when we consider ``practical'' cost functions mapping to $\RZero \cup \{ \infty \}$). It is however possible to avoid the constraints on uniform continuity and an upper bound and instead just require an l.s.c.\ cost function.

At the end of Section~\ref{KantRelax} we have generalized the existence results for (KP) to l.s.c.\ cost functions. In a similar fashion, we would like to relax the constraint on $c$ in Theorem~\ref{BoundedPDRel}. We first make the following observation for all l.s.c.\ cost functions bounded from below. Boundedness from below is usually a practical assumption for cost functions, because negative costs can often be avoided by reformulating the problem to only assign non-negative costs. After the lemma we can achieve strong duality for much more general settings.

\begin{lemma}[Adapted from\ {\cite[Lemma~1.41]{San2015}}]\label{ConvOfKanProb}
	Let $c$ and $c_n$ be l.s.c.\ and bounded from below for all \NinN. If $c_n \nearrow c$ for \Ninf, i.e. ${(c_n)}_{\NinN}$ converges increasingly towards $c$, then
	\[ \lim\limits_{\Ninf} \inf \left\{ \int c_n~\D[\gamma] : \gamma \in \TP{\mu}{\nu} \right\} = \inf \left\{ \int c~\D[\gamma] : \gamma \in \TP{\mu}{\nu} \right\}. \]
\end{lemma}

\begin{proof}
	This proof was adapted from the proof of\ \cite[Lemma~1.41]{San2015}.

	As $c_n \le c, c_n \le c_{n + 1}$ for all \NinN{} we have for all $\gamma \in \TP{\mu}{\nu}$:
	\[ \forall \NinN: \int\limits_{X \times Y} c_n~\D[\gamma] \le \int\limits_{X \times Y} c~\D[\gamma],\ \int\limits_{X \times Y} c_n~\D[\gamma] \le \int\limits_{X \times Y} c_{n + 1}~\D[\gamma] \]
	\[ \Rightarrow \forall \NinN: \inf\limits_{\gamma \in \TP{\mu}{\nu}} \int\limits_{X \times Y} c_n~\D[\gamma] \le \inf\limits_{\gamma \in \TP{\mu}{\nu}} \int\limits_{X \times Y} c_{n + 1}~\D[\gamma] \le \inf\limits_{\gamma \in \TP{\mu}{\nu}} \int\limits_{X \times Y} c~\D[\gamma]. \]
	With the convergence of $c_n$ we thus have $\lim\limits_{\Ninf} \inf\limits_{\gamma \in \TP{\mu}{\nu}} \int c_n~\D[\gamma] \le \inf\limits_{\gamma \in \TP{\mu}{\nu}} \int c~\D[\gamma]$.

	In the proof of Theorem~\ref{KPAdmitPolishLSC} we have already seen that every sequence from \TP{\mu}{\nu} is tight, and from the theorem itself we know that for every $c_n$ there exists an optimal transport map $\gamma_n$. Applying Prokhorov (\ref{Prok}) once more, we obtain a subsequence ${(\gamma_{n_k})}_{\NinN[k]}$ which weakly converges to a transport plan $\tilde{\gamma}$. Picking an arbitrary \NinN[j] we have $\forall n \ge j: c_n \ge c_j$ by induction from the monotonicity of our original sequence, and hence
	\[ \lim\limits_{\NinN}\,\inf\limits_{\gamma \in \TP{\mu}{\nu}} \int\limits_{X \times Y} c_n~\D[\gamma] = \lim\limits_{\NinN} \int\limits_{X \times Y} c_n~\D[\gamma_n] \ge \liminf\limits_{\NinN} \int\limits_{X \times Y} c_j~\D[\gamma_n]. \]
	As $\vert c_j \vert \le c$ we can now take the limit over $j$ and apply Lebesgue (\ref{Leb-DomCon}) to get
	\[ \lim\limits_{\NinN}\,\inf\limits_{\gamma \in \TP{\mu}{\nu}} \int\limits_{X \times Y} c_n~\D[\gamma] \ge \liminf\limits_{\NinN}\,\lim\limits_{\NinN[j]} \int\limits_{X \times Y} c_j~\D[\gamma_n] = \liminf\limits_{\NinN} \int\limits_{X \times Y} c~\D[\gamma_n]. \]
	We finally see that $\liminf\limits_{\NinN} \int c~\D[\gamma_n] \ge \inf\limits_{\gamma \in \TP{\mu}{\nu}} \int c~\D[\gamma]$, and thus our claim.
\end{proof}

With Lemma~\ref{ConvOfKanProb} we can obtain a theorem about strong duality with preliminaries resembling those of Theorem~\ref{KPAdmitPolishLSC}. This result will conclude the first part, having laid out a foundation of optimal transport theory. The focus will then turn towards regularization in optimal transport, applying analogous duality results, and estimating optimal transport plans and maps in large-scale problem settings.

\begin{theorem}[Adapted from\ {\cite[Theorem~1.42]{San2015}}]\label{GenPDRel}
	Let $X$ and $Y$ be complete and separable metric spaces, and $\map[c]{X \times Y}{\RZero \cup \{ \infty \}}$ be an l.s.c.\ cost function. Then, $\sup\!\text{\normalfont\,(DP)} = \inf\!\text{\normalfont\,(KP)}$ holds.
\end{theorem}

\begin{proof}
	This proof was adapted from the proof of\ \cite[Theorem~1.42]{San2015}.

	For this proof, we would like to apply Lemma~\ref{ConvOfKanProb}, however this requires the existence of an l.s.c.\ sequence ${(c_n)}_{\NinN}, \forall \NinN: \map[c_n]{X \times Y}{\RZero \cup \{ \infty \}}$ with $c_n \nearrow c$ for \Ninf. It turns out that this is no problem: as\ \cite[Box~1.5]{San2015} points out, $c$ is l.s.c.\ if and only if there exists a sequence of $K$-Lipschitz functions with $c_n \nearrow c$ for \Ninf. Here $K$-Lipschitz refers to $\forall \NinN{}, z, \hat{z} \in X \times Y: \vert c_n(z) - c_n(\hat{z}) \vert \le K \cdot d_{X \times Y}(z, \hat{z})$.

	With Lipschitz continuity implying uniform continuity, and by uniform continuity implying lower semi-continuity through regular continuity, this sequence is l.s.c. Using the uniform continuity and the bound obtained from $c$, we can apply Theorem~\ref{BoundedPDRel} and $\varphi \oplus \psi \le c_n \le c$ to get
	\[ \inf\limits_{\gamma \in \TP{\mu}{\nu}} \int\limits_{X \times Y} c_n~\D[\gamma] = \sup\limits_{\varphi \in \CF[b]{X}, \psi \in \CF[b]{Y}, \varphi \oplus \psi \le c_n} \int\limits_X \varphi~\D + \int\limits_Y \psi~\D[\nu] \]
	\[ \le \sup\limits_{\varphi \in \CF[b]{X}, \psi \in \CF[b]{Y}, \varphi \oplus \psi \le c} \int\limits_X \varphi~\D + \int\limits_Y \psi~\D[\nu]. \]

	Using Lemma~\ref{ConvOfKanProb} for \Ninf, we have $\inf \text{(KP)} \le \sup \text{(DP)}$. Hence, in combination with $\sup \text{(DP)} \le \inf \text{(KP)}$, strong duality holds.
\end{proof}

This concludes the first part of the thesis. We will now work towards large-scale mapping estimation, making use of regularized optimal transport along the way.